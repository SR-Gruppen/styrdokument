\section{Sektionsstyrelsen}

\subsection{Ansvarsområden}

\subsubsection{Det åligger sektionsstyrelsen att:}

\begin{itemize}
  \item verka för sammanhållningen mellan sektionsmedlemmarna och verka för deras gemensamma intressen 
  \item leda sektionens arbete 
  \item övervaka genomförandet av sektionsmötesbeslut och se till att de verkställs 
  \item framlägga budget med förslag på sektionsavgift till sektionsmötet 
  \item framlägga preliminär verksamhetsplan vid sista ordinarie vårmötet 
  \item lämna förslag på representanter till sektionens valberedning 
  \item utse representant från styrelsen till DatE-ITs styrelse
\end{itemize}

\subsubsection{Det åligger sektionsordförande att:} 

\begin{itemize}
  \item tillse att sektionens beslut verkställs 
  \item föra sektionens talan då något annat ej stadgats eller beslutats 
  \item teckna sektionens firma 
  \item leda och övervaka arbetet inom sektionsstyrelsen 
  \item representera sektionen på kårledningsutskottet
\end{itemize}

\subsubsection{Det åligger sektionens vice ordförande att:}

\begin{itemize}
  \item biträda ordföranden i dennes värv 
  \item i ordförandens frånvaro överta dennes åligganden 
  \item kontrollera så att arbetet sker i enighet med sektionens bestämmelser 
  \item representera sektionen på kårens nöjeslivsutskott
\end{itemize}

\subsubsection{Det åligger styrelsens kassör att:}

\begin{itemize}
  \item följa sina skyldigheter enligt det ekonomiska reglementet
  \item representera sektionen på kårens sektionsekonomiforum
\end{itemize}

\subsubsection{Det åligger styrelsens sekreterare att:}

\begin{itemize}
  \item föra protokoll vid styrelsens möten och tillse att protokoll från såväl styrelse- som sektionsmöten anslås
  \item tillse att material som inkommer till sektionen anslås eller på annat sätt förmedlas till berörda parter
\end{itemize}

\subsubsection{Det åligger sektionens SAMO att:}

\begin{itemize}
    \item verka för studenternas trivsel på sektionen.
    \item bistå studenter i frågor kring fysisk och psykisk hälsa
    \item föra sektionens talan i frågor kring psykosocial och fysisk studie- och arbetsmiljö.
    \item representera sektionen på kårens sociala utskott
\end{itemize}

\subsubsection{Det åligger samtliga ledamöter att:}

\begin{itemize}
  \item bistå styrelsen med information 
  \item aktivt deltaga i beslutsprocessen 
  \item redogöra för sin egen eller sin kommittés löpande verksamhet vid styrelsens möten 
\end{itemize}

\subsection{Insyn} 

Sektionsstyrelsen har full insyn i teknologsektionen alla organ och äger rätt att deltaga i deras möten med yttranderätt. 

\subsection{Medlemmar} 
Styrelsen består, utöver de i stadgarna uppräknade, även utav följande kommitéers ordförande. 
\begin{itemize}
  \item DRUST 
  \item DAG 
  \item Delta 
  \item D6 
  \item DNollK
  \item DNS
\end{itemize}

\subsection{Suppleant} 

Då övrig medlem ej kan närvara vid möte har denne rätt att utse en suppleant ur samma kommitté. 
\subsection{Omröstning} 

\begin{itemize}
  \item Röstning med fullmakt får ej ske. 
  \item Omröstning skall ske öppet. 
  \item Vid lika utfall äger mötesordförande utslagsröst. 
  \item Då flera förslags ställs mot varandra skall röstningsförfarandet fastslås innan omröstning påbörjas.
\end{itemize}

\subsection{Lekmannarevisorer}
Teknologsektionens lekmannarevisorer har rätt att medverka med närvaro-
och yttranderätt på styrelsemöten.
\newpage

