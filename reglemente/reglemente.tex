\documentclass[a4paper]{dtek}

\setcounter{secnumdepth}{5}
\title{Reglemente}
\date{Utkast: SR-Gruppen Ver 2}

% If you need additional packages just for stadga.tex, put them here:
%\usepackage{...}

\begin{document}
\begin{titlepage}
  \thispagestyle{empty} % no header/footer
  \vspace*{\fill}
  \begin{center}
    \includegraphics[width=250pt]{dteklogo_orange.pdf}\\[3em]
    {\Huge Datateknologsektionen}\\[3em]
    {\Huge \textbf{Reglemente}}\\[1em]
    Utkast: SR-Gruppen Ver 2\\[20em]
  \end{center}
  \vspace*{\fill}
\end{titlepage}

% After front page, we activate fancyhdr
\makeheadfoot

\setcounter{tocdepth}{2}
\tableofcontents

\section{Allmänt}
\begin{center}
  [Inga regleringar utöver Teknologsektionens stadga]
\end{center}
\section{Medlemmar}
\begin{center}
  [Inga regleringar utöver Teknologsektionens stadga]
\end{center}
\section{Organisation}
\begin{center}
  [Inga regleringar utöver Teknologsektionens stadga]
\end{center}
\section{Sektionsmötet}
\begin{center}
  [Inga regleringar utöver Teknologsektionens stadga]
\end{center}
\section{Valberedningen och personval}
\begin{center}
  [Inga regleringar utöver Teknologsektionens stadga]
\end{center}
\section{Sektionsstyrelsen}
\para[Ansvarsområden] Det åligger sektionsstyrelsen att:
\begin{itemize}
  \item verka för sammanhållningen mellan sektionsmedlemmarna och verka för deras gemensamma intressen
  \item leda sektionens arbete
  \item övervaka genomförandet av sektionsmötesbeslut och se till att de verkställs
  \item framlägga budget med förslag på sektionsavgift till sektionsmötet
  \item framlägga preliminär verksamhetsplan vid sista ordinarie vårmötet
  \item lämna förslag på representanter till sektionens valberedning
  \item utse representant från styrelsen till DatE-ITs styrelse
  \item lämna ett skriftligt bemötande till motioner som inkommit i tid senast 5 läsdagar innan sektionsmöte
\end{itemize}
\para Det åligger sektionsordförande att:
\begin{itemize}
  \item tillse att sektionens beslut verkställs
  \item föra sektionens talan då något annat ej stadgats eller beslutats
  \item teckna sektionens firma
  \item leda och övervaka arbetet inom sektionsstyrelsen
  \item representera sektionen på kårledningsutskottet
  \item till varje sektionsmöte kunna redogöra om sektionens verksamhet
  \item se till att ordförande i varje kommitté har tillgång till och kunskap omsektionens stadgar, reglemente och policies
  \item Se till att ordförande i varje kommitté skriver en verksamhetsrapport inför varje brytpunkt.
\end{itemize}
\para Det åligger sektionens vice ordförande att:
\begin{itemize}
  \item biträda ordföranden i dennes värv
  \item i ordförandens frånvaro överta dennes åligganden
  \item kontrollera så att arbetet sker i enighet med sektionens bestämmelser
  \item representera sektionen på kårens nöjeslivsutskott
\end{itemize}
\para Det åligger styrelsens kassör att:
\begin{itemize}
  \item teckna sektionens firma
  \item se till att kassör i varje kommitté har tillgång till och kunskap om sektionens stadgar, reglemente och policies
  \item Fortlöpande kontrollera sektionens räkenskaper och bokföring
  \item representera sektionen på kårens sektionsekonomiforum
  \item i samråd med styrelsen upprätta budgetförslag till första ordinarie höstmötet
  \item till varje sektionsmöte kunna redogöra för sektionens ekonomiska ställning
  \item utbilda nya förtroendevalda i hur sektionens bokförings och redovisningssystem skall användas
\end{itemize}
\para Det åligger styrelsens sekreterare att:
\begin{itemize}
  \item föra protokoll vid styrelsens möten och tillse att protokoll från såväl styrelse- som sektionsmöten anslås
  \item tillse att material som inkommer till sektionen anslås eller på annat sätt förmedlas till berörda parter
  \item tillse att sektionens styrdokument hålls uppdaterade i enlighet med sektionsmötes- och styrelsebeslut
\end{itemize}
\para Det åligger sektionens SAMO att:
\begin{itemize}
  \item verka för studenternas trivsel på sektionen.
  \item bistå studenter i frågor kring fysisk och psykisk hälsa
  \item föra sektionens talan i frågor kring psykosocial och fysisk studie- och arbetsmiljö.
  \item representera sektionen på kårens sociala utskott
\end{itemize}
\para Det åligger styrelsens övriga medlemmar att:
\begin{itemize}
  \item bistå styrelsen med information
  \item aktivt deltaga i beslutsprocessen
  \item redogöra för sin egen eller sin kommittés löpande verksamhet vid styrelsens möten
\end{itemize}
\section{Kommittéer}
\para Teknologsektionens kommittéer är:
\begin{itemize}
  \item DRust
  \item DAG
  \item Delta
  \item D6
  \item DNollK
  \item DMNollK
\end{itemize}
\para Ordförande i kommitté är skyldig att
\begin{itemize}
  \item kontinuerligt meddela kommitténs ekonomiska status till styret
  \item tillsammans med kommitténs kassör ansvara för att kommittén förvaltar sina tillgångar i enlighet med sektionens stadgar, reglementen och beslut.
  \item tillsammans med kommitténs kassör skriva ett bindande avtal med styret angående skuldfrågan vid felaktig bokföring
\end{itemize}
\para Kassör i kommitté är skyldig att
\begin{itemize}
  \item föra kassabok av sådan typ som godkänts av sektionens revisorer
  \item tillsammans med kommitténs ordförande ansvara för att kommittén förvaltar sina tillgångar i enlighet med sektionens stadgar, reglementen och beslut.
  \item tillsammans med kommitténs ordförande skriva ett bindandeavtal med Styret angående skuldfrågan vid felaktigt förd bokföring
  \item på varje sektionsmöte redovisa kommitténs ekonomiska situation
  \item arkivera kommitténs bokföring, på en plats anvisad av Styret, så lång tid som föreskrivs för den organisationsform som datateknologsektionen är.
  \item lägga en budget
\end{itemize}
\para[DRust] DRust har i uppdrag att ansvara för teknologsektionens lokaler.
\para DRust består av ordförande, kassör och 4 övriga ledamöter.
\para Kassör i DRust har i uppdrag att överse DBus verksamhet.
\para[DAG] DAG har i uppdrag att verka för samverkan mellan datateknologsektionen och arbetsmarknaden.
\para DAG består av ordförande, kassör och 5 övriga ledamöter.
\para[Delta] Delta har i uppdrag att verka för sektionsfrämjande aktiviteter.
\para Delta består av ordförande, kassör och 4 övriga ledamöter.
\para[D6] D6 har i uppdrag att arrangera fester.
\para Ordförande i D6 ska representera sektionen på kårens gasqueforum.
\para D6 består av ordförande, kassör och 6 övriga ledamöter.
\para[DNollK] DNollK har i uppdrag att sköta mottagningen.
\para Ordförande i DNollK ska representera sektionen på kårens mottagningsforum.
\para DNollK består av ordförande, kassör och 6 övriga ledamöter.
\para Mandatperioden för DNollK är 1:a januari – 31:a december.
\para[DMNollK] DMNollK har i uppdrag att sköta mastermottagningen.
\para DMNollK består av ordförande, kassör och 4–5 övriga ledamöter.
\para Mandatperioden för DNollK är 1:a januari – 31:a december.
\section{Utskott}
\para Teknologsektionens utskott är:
\begin{itemize}
  \item DLirium
  \item dHack
  \item Ståthållarämbetet
  \item JämställD
  \item DFoto
  \item DKock
  \item DBus
\end{itemize}
\para[DLirium] Ska roa och kritiskt granska sektionen, kåren och Chalmers.
\para DLirium består av 1–4 redaktörer. Ordförande, tillika chefsredaktör och ansvarig utgivare väljs internt och rapporteras till styrelsen.
\para[dHack] Ska driva teknologsektionens IT-tjänster och datorsystem, samt främja hackerandan på sektionen.
\para dHack består av en ordförande och 0–3 övriga ledamöter.
\para[Ståthållarämbetet] Se efter d08gong med tillbehör samt sektionens fana, representera sektionen på traditionsenliga arrangemang, arrangera sektionens årliga firande av Hackes födelsedag. och andra högtidliga arrangemang samt hålla alumnkontakt.
\para Ståthållarämbetet består av fanbärare tillika ordförande, ceremonimästare och 0–4 vapendragare.
\para[JämställD] JämställD ska främja jämlikhet och välmående på datateknologsektionen.
\para JämställD består av ordförande, Teknologsektionens SAMO och 2–4 övriga ledamöter.
\para[DFoto] DFoto har i uppdrag att i bild dokumentera teknologsektionens verksamhet.
\para DFoto består av ordförande och 0–5 övriga ledamöter
\para[DKock] DKock ska främja datateknologers matlagningsintresse samt vid behov hjälpa övriga kommittéer med matlagning.
\para DKock består av ordförande och 0–5 övriga ledamöter
\para[DBus] DBus har i uppdrag att förvalta sektionens fordon och allt som har samröre med detta.
\para DBus består av ordförande tillika bilansvarig och vice bilansvarig. Valbar till bilansvarig och vice bilansvarig är den som innehar giltigt B-körkort. Vid avsaknande av ordförande faller DBUS ansvar på styrelsen.
\section{Intresseföreningar}
Teknologsektionens intresseföreningar är:
\begin{itemize}
  \item Datas tyska mousserande vin
  \item DAF
  \item Datas Ludologer
  \item plantaD
  \item bakaD
  \item Datas Sommelierliga
  \item Datas tyska mousserande vin
  \item iDrott
\end{itemize}
\section{Teknologsektionens identitet}
\begin{center}
  [Inga regleringar utöver Teknologsektionens stadga]
\end{center}
\section{Protokoll och anslagning}
\begin{center}
  [Inga regleringar utöver Teknologsektionens stadga]
\end{center}
\section{Revision och ansvarsfrihet}
\begin{center}
  [Inga regleringar utöver Teknologsektionens stadga]
\end{center}
\section{Teknologsektionens upplösning}
\begin{center}
  [Inga regleringar utöver Teknologsektionens stadga]
\end{center}
\section{Ändrings- och tolkningsfrågor}
\begin{center}
  [Inga regleringar utöver Teknologsektionens stadga]
\end{center}
\section{Inspektor}
\begin{center}
  [Inga regleringar utöver Teknologsektionens stadga]
\end{center}
\section{Hedersbetygelser}
\begin{center}
  [Inga regleringar utöver Teknologsektionens stadga]
\end{center}

\end{document}
