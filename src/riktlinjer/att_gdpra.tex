\documentclass[a4paper]{dtek}

\setcounter{secnumdepth}{5}
\title{Att GDPRA}
\date{Utkast: SR-Gruppen Ver 2}

\begin{document}
\section{Personuppgiftspolicy för uppdragstagare}
Använd den här policyn som en guide för att hjälpa dig i ditt GDPR-arbete. Om du har
några frågor eller funderingar, kontakta Styrelsen: Styrelsen@dtek.se.
Den här policyn skapades 2018 och uppdaterades 2020 som ett led i arbetet med att göra
sektionen, dess kommittéer och intresseföreningar kompatibla med GDPR och ska om allt
fungerar som det är tänkt uppdateras löpande.

\subsection{Om GDPR och ideellt engagemang}
Det är viktigt att det är lätt och roligt att engagera sig ideellt på datateknologsektionen,
även när vi behöver hålla oss till gällande lagar och regler. Därför försöker vi hålla admi-
nistrationen kring GDPR till ett minimum. Följande behöver dock utföras för att vi ska
följa lagen:

\subsection{Vid inval}
Den nyinvalde ska så snart som möjligt skriva på ett kontrakt för att bli personuppgifts-
behandlare, vilket innebär att den får samla in och hantera personuppgifter för Datatek-
nologsektionen.

\subsection{Vid inaktivitet}
Om en person slutar vara aktiv i en kommitté eller intresseförening förlorar den åtkomst
till kommitténs personuppgifter tills den blir aktiv igen, detta för att minimera antalet
personer med tillgång till personuppgifter.

\subsection{Tredje parts åtkomst}
När personuppgifter delas med parter som ej är personuppgiftsbehandlare på Datatekno-
logsektionen är de att anse som tredje part. Det förekommer att personuppgifter delas med
tredje parter i olika delar av sektionens verksamhet.

\subsubsection{Huvudregel}
En tredje part kan få åtkomst till personuppgifter om det föreligger särskilda skäl, förutsatt
att detta godkänns av personuppgiftsansvarige i Styrelsen. Vid sådana omständigheter ska
personen i fråga även underteckna ovan nämnda kontrakt.

%5.4.1 borde hrefas
\subsubsection{Undantag}
Undantag från 5.4.1 är då personuppgiftsbehandlare nyttjar tredje parts tjänster på inter-
net. Då skall personuppgiftsbehandlaren försäkra att tjänsten har en personuppgiftspolicy som är enligt med GDPR och ej delar sektionens personuppgifter med annan tredje part
som ej uppfyller huvud- eller undantagsregeln.

Vid samarrangemang bör en personuppgiftsansvarig utses i den sektionskommittée som ar-
rangerar med en tredje part. Dess uppgift är att säkerställa att den arrangerade kommittén
hanterar personuppgifter som sektionen anser som lämpligt för att undantaget skall kunna
tillämpas.

\subsection{Dokumentation}
Styrelsen ska upprätthålla en lista över alla personer med åtkomst till personuppgifter,
samt i vilka sammanhang de kommer åt dessa personuppgifter. Exempel på sammanhang
är \enquote{Phixare i D6}.

\subsection{Behandling av personuppgifter}
GDPR är hårdare kring lagring av personuppgifter än den tidigare personuppgiftslagen
(PUL). Nedan följer praktiska riktlinjer för hur personuppgifter ska hanteras med nya
lagstiftningen.

\subsubsection{Grundprinciper}
GDPRs grundprinciper är att man som personuppgiftsansvarig:
\begin{itemize}
    \item Måste ha stöd i dataskyddsförordningen för att få behandla personuppgifter.
    \item Bara får samla in personuppgifter för specifika, särskilt angivna och berättigade ändamål.
    \item Inte ska behandla fler personuppgifter än vad som behövs för ändamålen.
    \item Ska se till att personuppgifterna är riktiga.
    \item Ska radera personuppgifterna när de inte längre behövs.
    \item Ska skydda personuppgifterna, till exempel så att inte. obehöriga får tillgång till dem och så att de inte förloras eller förstörs.
    \item Ska kunna visa att och hur ni lever upp till dataskyddsförordningen.
\end{itemize}

Exempel på personuppgifter:
\begin{itemize}
    \item Ett namn och efternamn
    \item En hemadress
    \item En e-postadress såsom namn.efternamn@företag.com
    \item Ett id-kortsnummer
    \item Platsinformation (t.ex. platsfunktionen på en mobiltelefon)
    \item En IP-adress
    \item Kakor
    \item Reklamidentifieraren på din telefon
    \item Uppgifter som innehas av ett sjukhus eller en läkare och som skulle kunna vara en symbol som fungerar som en unik identifikation av en person
\end{itemize}

\subsubsection{Opt-in – aktivt medgivande}
GDPR kräver aktivt medgivande vilket innebär att varje person vars personuppgifter ska
lagras ger aktivt samtycke till detta. Det innebär att de vars personuppgifter vi lagrar
måste ge aktivt samtycke \textit{innan} vi lagrar deras personuppgifter och att hen kan anses
förstå innebörden av datalagringen.

Det räcker inte med att begära ett godkännande för att spara personuppgifterna utan man behöver även förklara varför och hur länge och hur de kommer att behandlas.

Några exempel:
\begin{itemize}
    \item På anmälan till ett evenemang behöver det finnas en obligatorisk kryssruta där den
som anmäler sig godkänner att dess personuppgifter sparas och behandlas av sektionen för att arrangemanget skall kunna genomföras eller för att personen skall kunna
ta del av arrangemanget. Ytterligare skall personen informeras om varför och hur
personuppgifterna behandlas och sparas. Exempel på text:

\textit{\textbf{För att arrangemanget ska gå att planera och genomföra behöver vi spara dina ovan
angivna uppgifter under en begränsad tid. Dina personuppgifter kan komma att delas med
tredje part, t.ex. puffar eller samarrangörer, om det behövs för arrangemanget. Strax efter
arrangemanget kommer uppgifterna att raderas.
}}


    \item På puff-formulär vill spara arbetarnas uppgifter en längre tid än till strax efter arrangemangets slut. Anledningar till detta kan vara att att man vill bjuda arbetarna på ett tackkalas eller be om hjälp vid senare tillfälle. Då kan följande stå:
    
\textit{\textbf{Dina uppgifter kommer att sparas för att vi ska kunna kontakta dig även vid senare tillfälle, t.ex. om vi behöver hjälp vid något annat arrangemang eller på något sätt vill tacka dig för din insats. Dina uppgifter kommer raderas vid slutet av vårt verksamhetsår.}}
        
\end{itemize}

\subsubsection{Tidsbestämt}
Uppgifter får inte sparas längre än nödvändigt. Några exempel:
\begin{itemize}
    \item Anmälningslistor till sittningar skall raderas senast två veckor efter sittningen.
    \item Pufflistor skall raderas senast två veckor efter tackkalaset.
\end{itemize}

\subsubsection{Specificitet}
En personuppgift får bara användas till det som den som äger personuppgiften har godkänt
att den får användas till.

Antag att det finns ett godkännande för att spara e-postadresser för att kunna kontakta
personer. Då är det endast okej för detta syfte och man får då inte använda dessa uppgifter för att skapa en mailinglista till alla som går på data eller dela uppgifterna till tredje part.

\subsubsection{Spara alltid hela namnet}
Fullständigt namn ska alltid sparas tillsammans med andra personuppgifter för en per-
son så att personuppgiftsansvarige har en rimlig möjlighet att t.ex. ta bort en persons
personuppgifter om den vill bli bortglömd.

Det kan vara svårt att få deltagare att fylla i sina namn. Därför rekommenderar vi att
fullständigt namn och smeknamn samlas in som separata fält för att uppmuntra de som
fyller i listan eller formuläret att ge användbar information.

\subsection{Shared Drive}
Shared Drive är som en vanlig My Drive men det är gruppen och inte individerna som äger
innehållet. Varje kommitté och förening får ha en och endast en TeamDrive som de ärver
från tidigare år. För dessa gäller:
\begin{itemize}
    \item Högst upp i mapphierarkin bör det finnas en mapp för varje år, t.ex. “2017” och “2018”.
    \item Alla mappar och dokument döps så att en utomstående förstår vad det handlar om. Olämpliga namn kan vara “Supa satan mapp” och “Lite roligt bös”. Döp dessa istället till “Postom” och “Förslag på arr” så följer ni policyn och gör livet lättare för era efterträdare.
    \item Alla dokument och mappar som innehåller personuppgifter ska ha ett namn med
suffixet “[PU]” (PersonUppgifter). T.ex. mappen “Möten[PU]” eller filen "Kontaktuppgifter sittande[PU]”.
\end{itemize}

Det är inte allt för sällan som man har ett gemensamt arrangemang med andra kommittéer
eller puffar som behöver ta del av ett eller flera dokument.

\begin{itemize}
    \item Om det är puffar som klarar sig med enstaka dokument, dela då endast dessa.
    \item Om det är ett samarr och utomstående/andra kommittéer behöver tillgång till alla filer är det okej att skapa en temporär Shared Drive dit man bjuder in alla arrangörer som behöver tillgång till filerna. Denna Shared Driven ska raderas alternativt ska allt innehåll flyttas efter arrangemanget om inte speciella omständigheter råder. Personuppgiftsansvarige beslutar om det råder speciella omtändigheter. Ovanstående regler gäller även för dessa.
\end{itemize}

\textbf{Tips:} Om det är en hel grupp som ska ha tillgång till en drive räcker det med att gruppen bjuds in för att alla ska få tillgång.

\textbf{OBS!}Tänk på att ni måste ha godkännande att dela till tredje part, se Opt-in ovan. Intresseföreningar och kommittéer på data räknas inte som tredje part eftersom de är en del av datateknologsektionen.

\begin{itemize}
    \item En mapp för varje år. T.ex. “2017”, “2018”.
    \item Det är okej att ha en mapp för alla år tidigare. T.ex. “1337–2016”
\end{itemize}

Vi rekommenderar att man följer ovanstående struktur för att göra det lätt för nya att
snabbt få en överblick över vilka dokument som finns och var de ligger, men om den
strukturen inte fungerar, använd en som fungerar bättre för er.

\subsection{GMail}
Använd enbart e-postadresserna du har genom sektionen till sektionsarbete, det är alltså
både otillåtet och dumt att använda den för privat bruk då du lämnar över adressen till
din efterträdare efter ditt verksamhetsår.

E-post får ej vidarebefordras till privata e-postadresser.

\textbf{Tips:} Sätt en signatur som automatiskt infogas i slutet av dina mejl. Förslag:\\
\textbf{\textit{Vänliga hälsningar,\\
<Förening> <Post> 20xx/20xx\\
Datateknologsektionen Chalmers Studentkår\\
www.dtek.se<http://www.dtek.se/>}}

\subsection{Kalender}
Man får gärna skapa en eller flera kalendrar. Viktigt är att tänka på sätta rätt åtkomsträttigheter och dela med rätt personer. Till exempel kan en bokningskalender som “databussen” eller “Basenpax”, ha följande inställningar:
\begin{itemize}
    \item Tillgänglig för alla: se endast ledig/upptagen (dölj uppgifter).
    \item Delad med: dbus@dtek.se respektive alla@dtek.se. Rätt att göra ändringar och hantera delning.
\end{itemize}

En intern kalender bör endast vara delad med kommittén och en kalender som delas med
utomstående ska inte innehålla personuppgifter som de kan se.

Man kan lägga till hela grupper i en kalender, men då måste varje medlem lägga till
kalendern via en länk som skickas till mejlen. Några bra kalendrar som finns som kan vara
bra för alla datateknologer är:

\begin{itemize}
    \item \textbf{Dtek-kalendern:} \\
    \href{https://calendar.google.com/calendar/embed?src=dtek.se\_0tavt7qtqphv86l4stb0aj3j88\%40group.calendar.google.com\&ctz=Europe\%2FStockholm}{https://calendar.google.com/calendar/embed?src=dtek.se\_0tavt7qtqphv86l4stb0aj3j8\\8\%40group.calendar.google.com\&ctz=Europe\%2FStockholm}

    \item \textbf{Basenpaxningar:}\\
    \href{https://calendar.google.com/calendar/embed?src=dtek.se\_b3sv1v3upmtjquppg10hhe59e0\%40group.calendar.google.com\&ctz=Europe\%2FStockholm}{https://calendar.google.com/calendar/embed?src=dtek.se\_b3sv1v3upmtjquppg10hhe5\\9e0\%40group.calendar.google.com\&ctz=Europe\%2FStockholm}

    \item \textbf{Databussen:}\\
    \href{https://calendar.google.com/calendar/embed?src=dtek.se\_69sdfhe5527mh3u9tk9146imak\%40group.calendar.google.com\&ctz=Europe\%2FStockholm}{https://calendar.google.com/calendar/embed?src=dtek.se\_69sdfhe5527mh3u9tk9146i\\mak\%40group.calendar.google.com\&ctz=Europe\%2FStockholm}

\end{itemize}

\subsection{Handlingsplan vid dataläcka}
Vid minsta misstanke om dataläcka, kontakta omedelbart personuppgiftsansvarig i Styrel-
sen: Styrelsen@dtek.se.

\end{document}