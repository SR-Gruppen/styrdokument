\documentclass[a4paper]{dtek}

\setcounter{secnumdepth}{5}
\title{Att arrangera}
\date{Utkast: SR-Gruppen v2026.1-pre3}

\makeheadfoot

\begin{document}
\section{Arrangemangspolicy}
Denna policy gäller alla arrangemang i Datateknologsektionens regi, samt övriga arrangemang i de lokaler som disponeras av Datateknologsektionen.

\subsection{Omfattning}
Alla arrangemang skall vara inkluderande för alla medlemmar på sektionen med ett fåtal undantag, som till exempel: phaddersittningar, mottagningsarrangemang och kommittetradtionella arrangemang. Arrangemang med en specifik målgrupp ska ha styrelsens godkännande.

\subsection{Hållbarhet}
Alla arrangemang ska sikta mot att minimera sin miljö- och klimatpåverkan.

Några exempel på sätt att minimera sin miljöpåverkan kan vara att:
\begin{itemize}
    \item I största utsträckning använda återanvändbara tallrikar, bestick och glas.
    \item Källsortera rester från arrangemanget.
    \item I största mån köpa svenska och ekologiska råvaror.
\end{itemize}

\subsection{Allergier}
Om ett evenemang har föranmälan ska alla med anmälda allergier få likvärdig mat som övriga i den utsträckning det går. Om ingen föranmälan sker, t.ex. afterschool och pubar, ska veganskt och glutenfritt alternativ kunna erbjudas. Undantag kan ges vid t.ex. provningar.

\subsection{Alkohol och droger}
Arrangörer ska arrangera i enlighet med Chalmers Studentkårs policy \textit{Fest- och alkoholpo-
licy vid Chalmers Studentkår.}

\section{Basenpolicy}
Alla medlemmar i Datateknologsektionen har fri tillgång till sektionslokalen Basen. Därför har en policy för användande av Basen tagits fram.

\subsection{Nyttjande av lokalen}
Vid nyttjande av lokalen råder skötsel-, ordning- och städbestämmelser.

\subsubsection{Skyldigheter}
Medlem skall
\begin{itemize}
    \item följa sektionens och kårens styrdokument, samt svensk lag.
    \item Städa efter sig själv.
    \item ej störa andra.
    \item ej avlägsna objekt från Basen utan styrelsens godkännande.
    \item ej förvara stora och klumpiga saker i Basen eller dess närhet.
    \item medverka i Nollstäd.
    \item ansvara för sina gäster.
\end{itemize}

\subsubsection{Undantag}
\begin{itemize}
    \item Vid arrangemang begränsas ytan för gemene teknolog till köket. Vid vissa arrangemang är även vistelse i köket förbjudet, exempelvis under serveringstillstånd.
    \item Under längre skoluppehåll är Basen avstängd, då resurser inte finns för att städa den.
    \item Delar av, eller hela Basen kan efter styrelsebeslut stängas ner av andra anledningar.
\end{itemize}

\subsection{Fest i lokalen}
Alla fester i Basen skall följa nedanstående regler. Det räknas som fest när fyra eller fler dricker alkohol samtidigt i Basen. Ansvariga har rätt till att avvisa folk från lokalen som inte följer dessa regler. Datateknologer har dock alltid rätt att få tillgång till Basens kök.
\subsubsection{Regler för fest i Basen}
\begin{itemize}
    \item En ansvarig ska finnas på plats, som alltid hålla sig vid sina sinnens fulla bruk och ha uppsikt över hela lokalen och dess omgivning.
    \item Man får ej bruka alkohol i Basen mellan 03:00 och 17:00 på helgfria vardagar.
    \item Det är bara sektionsmedlemmar som kan vara ansvariga.
    \item Köket skall vara fritt från fest.
    \item Brandgränsen skall alltid följas(70 personer i \textbf{hela} lokalen).
    \item Efter 22:00 ska fönsterna vara stängda om nästkommande dag är en vardag.
\end{itemize}

\subsubsection{Paxa basen för arr}
För att utföra ett planerat arr i Basen så ska man skicka in en in paxning via Basenpax formuläret. Beslut om paxningar tas upp nästkommande styrelsemöte.
Efter godkännade från styrelsen så ska arrangemanget aktivitetanmälas på studentportalen. Där ska även den ansvariga skrivas upp.

\subsubsection{Spontan fest}
Om du öppnar upp för fest i basen ska detta anmälas direkt vilket man gör via fest@dtek.se.
I mailet anger man vem som står som ansvarig. Om ni känner er osäkra på om en anmälan
krävs, gör det hellre en gång för mycket än för lite då Styrelsen och DRust vill vilka som
vistats i Basen i fall något händer. 
När festen är klar behöver man städa Basen enligt arrstädslappen.

\subsubsection{Spontanöppning av flipperrummet}
Flipperrummet kan spontanöppnas för fest om Basen inte är bokad och inte redan är öppnad.
Detta görs genom att maila vem som är ansvarig till fest@dtek.se, på samma sätt som när man öppnar hela Basen.
Efteråt ska man städa flipper enligt flipperstädlappen.

\subsubsection{Överlåta festen}
Om du öppnar basen är du första ansvarig att se till att det blir städat. Om du överlåter
festen till någon annan måste personen som tar över ansvaret skicka in ett nytt mail till
fest@dtek.se där den anger att den tar över.

\subsubsection{Städ}
Om man har arrangerat i Basen så ska det städas enligt de instruktioner som finns på
arrstäd-lapparna hängandes i Basens städskrubb. Ifall att någon annan varit i köket och
har lagat mat eller arrangerat skall detta anges, annars är det ert ansvar att det är städat.
Ett mail ska skickas till fest@dtek.se där man säger till att Basen stängs ner, samt bifogar
bilder från Puben, Köket, Hallen och Flipperrummet. Dessa är till för att skydda dig som
ansvarig från skuld. Städningen ska vara avklarade innan 08:00 följande morgon. Om det
inte är städat innan dess eller om DRust alternativt Styret anser att du inte har städat
eller fyllt i hela listan kommer städningen inte att godkännas och kan leda till påföljder.

\subsection{Åtgärder}
Om några regler bryts kommer åtgärder att vidtas. Du som ansvarig kommer att bli med-
delad att ärendet kommer lyftas på ett styrelsemöte. Styrelsen har rätt att bestämma vilka
åtgärder som kommer tas.


\section{Aspningspolicy}
Alla kommittéer på sektionen har rätt till att hålla i en aspning. Tanken med aspningen är att visa asparna vad kommitténs arbete går ut på och ge dem en chans att testa på arbetet. Fokus ska ligga på verksamheten. För att se till att alla kommittéer har samma förutsättningar och att alla aspar behandlas på ett värdigt sätt har denna aspningspolicy tagits fram. 

\subsection{Innan Aspningen}
\begin{itemize}
    \item Eftersom det är väldigt många kommittéer som arrangerar aspning samtidigt är det viktigt att sätta sitt aspschema så tidigt som möjligt. Det kommer vara omöjligt att helt undvika dubbelbokningar men se till att undvika det i största mån. 
    \item Innan aspningen börjar är det kul med aspaffischer, dessa kan sättas upp i Basen men styrelsen har rätt att plocka ner aspaffischer som anses vara opassande. Riktlinjer för aspaffischer är
    \begin{itemize}
        \item Alkohol får inte vara fokus av affischen.
        \item Affischer får inte innehålla något form av kränkande material. 
    \end{itemize}
\end{itemize}
\subsubsection{Aspningsplan}
Innan aspningsperioden böjar ska kommittéer skriva ner en aspningsplan som innehåller arr som ska hållas, en beskrivning av dem och en ungefärlig kostnadsberäkning och sedan skicka detta till styrelsen. 


\subsection{Under Aspningen}
\subsubsection{Förhållningsregler}
\begin{itemize}
    \item Det är inte tillåtet att lyfta sin egen kommitté på bekostnad av en annan kommitté. 
    \item Det är inte tillåtet att avråda aspar från att aspa specifika kommittéer. 
    \item Aspningstillfällen skall marknadsföras i enlighet med sektionens kommunikationspolicy, och alla skall uppmuntras lika mycket till att söka eller aspa.
    \item Ingen alkoholhets får förekomma. 
    \item Om alkohol erbjuds vid ett aspningsarrangemang så skall även motsvarande alkoholfria alternativ erbjudas.
    \item Arrangörer har rätt att avvisa aspar som missköter sig från pågående asparrangemang, men endast styrelsen har rätt att stänga av aspar från aspningen i helhet.
    \item Person som arrangerar aspning eller medverkar på aspning av annan anledning än att aspa får ej missbruka sin maktposition på något sätt.
    \item Person som arrangerar aspning ska avstå från att inleda intim relation med asp.
    \item Aspar får inte favoriseras eller särbehandlas.
    \item Aspplagg ska vara roliga att ha på sig och inte vara förnedrande för den som bär det. 
\end{itemize}
\subsubsection{Arrangemang}
\begin{itemize}
    \item Vid varje arrangemang ska det finnas minst två stycken nyktra arrangerande som har ansvar. 
    \item Platsbegränsade arrangemang bör undvikas då alla ska ha möjlighet att delta under aspningen.
    \item Asptillfällen ska vara så billiga som möjligt att delta på, men det är okej för aspar att betala en mindre summa för mat om det skulle behövas. 
    \item Det ska meddelas i förväg ifall mat serveras eller ej under ett asptillfälle. 
    \item Asptillfällen ska ej ha fokus på alkohol, utan ska fokusera på kommitténs verksamhet.
\end{itemize}


\subsection{Efter Aspningen}
Efter aspningen är slut ska alla aspar fortsatt behandlas lika och ingen favorisering får förekomma även efter nominering från valberedningen. 
\subsubsection{Valberedning}
\begin{itemize}
    \item Varje kommitté har rätt att be om valberedning, valberedningens sammansättning beskrivs i reglementet. 
    \item Firande av/festande med nominerad grupp i egenskap av nominering får ej förekomma. 
    \item Det ska vara tydligt för de icke-nominerade att de fortfarande har möjlighet att söka kommittén på sektionsmötet. 
\end{itemize}
\end{document}