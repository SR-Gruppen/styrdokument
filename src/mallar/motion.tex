\documentclass[a4paper]{dtekmotion}


\title{Motion: Införande av obligatorisk hatt vid sektionens möten}

\makehf

\begin{document}
\maketitle

\section{Bakgrund}
Under de senaste mötena har det blivit uppenbart att en ny, oväntad ingrediens saknas i vår annars så seriösa vardag – nämligen hattar. I en tid av strikta regler och byråkratiska processer vill vi införa en liten dos humor och kreativitet. En hatt är inte bara en prydnad, utan ett uttryck för individualitet och samhörighet, och kan samtidigt fungera som en effektiv isbrytare. Denna trattmotion är skriven med glimten i ögat, som ett skämt, men med en underliggande poäng om att lite lekfullhet kan lyfta hela sektionens atmosfär.

\subsection*{Hattens Symbolik}
En hatt representerar mer än bara mode – den är en symbol för mod, originalitet och samhörighet. Genom att införa en obligatorisk hattpolicy vill vi ge varje möte en färgglad prägel, där allvaret får ge vika för skratt och kreativitet. Detta initiativ är en traktmotion, framtagen i ren glädjens anda, med tanken att våra möten ska bli minst lika minnesvärda som de är effektiva.

\section{Yrkande:}
Med ovanstående som bakgrund yrkar vi på: 

\begin{itemize}

\item \textbf{att} i sektionens stadga införa en ny bestämmelse under avsnittet \textit{Mötesklädsel} som lyder:
\begin{verbatim}
Vid samtliga sektionens möten skall deltagarna, utan undantag,
bära en hatt. Hatten får vara av valfri modell, färg och stil,
förutsatt att den bärs med stolthet och anses utgöra en del
av mötets officiella klädkod.
\end{verbatim}

\end{itemize}

\textit{Kalle \enquote{Hattmakaren} Hattson, d98,\\}
\textit{Lisa \enquote{Loftet} Lång, d86}

\end{document}
