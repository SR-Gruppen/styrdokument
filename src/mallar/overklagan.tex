\documentclass{dtek}

\title{Överklagansformulär}
\subtitle{För åtgärd till incident}
\date{UTKAST: SR-Gruppen Ver 2}

\begin{document}
\makeheadfoot
\maketitle

\begin{Form}

\paragraph{Allmänt}

Detta överklagansformulär skall undertecknas av överklagande part. Överklagande part kommer att representera ärendet på Kåren.

\paragraph{Överklagande}

Fullständigt namn och kontaktuppgifter.

\begin{tabular}{p{0.5\textwidth} p{0.5\textwidth}}
\TextField[name=ov_namn,width=\linewidth,borderstyle=U,borderwidth=0.5]{} & \TextField[name=ov_mobil,width=\linewidth,borderstyle=U,borderwidth=0.5]{} \\
Namn & Mobiltelefon\\[1ex]
\TextField[name=ov_epost,width=\linewidth,borderstyle=U,borderwidth=0.5]{} & \\
E-post &
\end{tabular}

\paragraph{Ärende}~\\[1ex]
\begin{tabular}{p{\textwidth}}
\TextField[name=ov_arende,width=\linewidth,borderstyle=U,borderwidth=0.5, multiline=true]{}\\[1ex]
Ärende som överklagan avser\\[1ex]
\TextField[name=ov_personer,width=\linewidth,borderstyle=U,borderwidth=0.5, multiline=true]{}\\[1ex]
Fullständigt namn på person(er) som överklagan avser och berör
\end{tabular}

\paragraph{Motivering}~\\
Motivera varför du anser att beslutet bör överklagas. Var gärna så specifik och utförlig du kan för att underlätta behandlingen.\\[1ex]
\begin{tabular}{p{\textwidth}}
\TextField[name=ov_motivering,width=\linewidth,borderstyle=U,borderwidth=0.5, multiline=true]{}
\end{tabular}

\newpage

\paragraph{Medgivande}~\\
Följande personer stödjer överklagan. Överklagande part representerar dessa i behandlingen av ärendet.\\[1ex]
\begin{tabular}{l p{0.35\textwidth} p{0.4\textwidth} l}
 & Fullständigt namn & Telefon och e-post & Signatur\\[1ex]
1. & \TextField[name=med_namn1,width=\linewidth,borderstyle=U,borderwidth=0.5]{} & \TextField[name=med_contact1,width=\linewidth,borderstyle=U,borderwidth=0.5]{} & \TextField[name=med_sign1,width=\linewidth,borderstyle=U,borderwidth=0.5]{}\\[1ex]
2. & \TextField[name=med_namn2,width=\linewidth,borderstyle=U,borderwidth=0.5]{} & \TextField[name=med_contact2,width=\linewidth,borderstyle=U,borderwidth=0.5]{} & \TextField[name=med_sign2,width=\linewidth,borderstyle=U,borderwidth=0.5]{}\\[1ex]
3. & \TextField[name=med_namn3,width=\linewidth,borderstyle=U,borderwidth=0.5]{} & \TextField[name=med_contact3,width=\linewidth,borderstyle=U,borderwidth=0.5]{} & \TextField[name=med_sign3,width=\linewidth,borderstyle=U,borderwidth=0.5]{}\\[1ex]
4. & \TextField[name=med_namn4,width=\linewidth,borderstyle=U,borderwidth=0.5]{} & \TextField[name=med_contact4,width=\linewidth,borderstyle=U,borderwidth=0.5]{} & \TextField[name=med_sign4,width=\linewidth,borderstyle=U,borderwidth=0.5]{}\\[1ex]
5. & \TextField[name=med_namn5,width=\linewidth,borderstyle=U,borderwidth=0.5]{} & \TextField[name=med_contact5,width=\linewidth,borderstyle=U,borderwidth=0.5]{} & \TextField[name=med_sign5,width=\linewidth,borderstyle=U,borderwidth=0.5]{}\\[1ex]
6. & \TextField[name=med_namn6,width=\linewidth,borderstyle=U,borderwidth=0.5]{} & \TextField[name=med_contact6,width=\linewidth,borderstyle=U,borderwidth=0.5]{} & \TextField[name=med_sign6,width=\linewidth,borderstyle=U,borderwidth=0.5]{}\\[1ex]
7. & \TextField[name=med_namn7,width=\linewidth,borderstyle=U,borderwidth=0.5]{} & \TextField[name=med_contact7,width=\linewidth,borderstyle=U,borderwidth=0.5]{} & \TextField[name=med_sign7,width=\linewidth,borderstyle=U,borderwidth=0.5]{}\\[1ex]
8. & \TextField[name=med_namn8,width=\linewidth,borderstyle=U,borderwidth=0.5]{} & \TextField[name=med_contact8,width=\linewidth,borderstyle=U,borderwidth=0.5]{} & \TextField[name=med_sign8,width=\linewidth,borderstyle=U,borderwidth=0.5]{}\\[1ex]
9. & \TextField[name=med_namn9,width=\linewidth,borderstyle=U,borderwidth=0.5]{} & \TextField[name=med_contact9,width=\linewidth,borderstyle=U,borderwidth=0.5]{} & \TextField[name=med_sign9,width=\linewidth,borderstyle=U,borderwidth=0.5]{}\\[1ex]
10.& \TextField[name=med_namn10,width=\linewidth,borderstyle=U,borderwidth=0.5]{} & \TextField[name=med_contact10,width=\linewidth,borderstyle=U,borderwidth=0.5]{} & \TextField[name=med_sign10,width=\linewidth,borderstyle=U,borderwidth=0.5]{}\\[1ex]
\end{tabular}
\vfill
\begin{tabular}{p{6cm}}
Göteborg den \TextField[name=ov_date,width=\linewidth,borderstyle=U,borderwidth=0.5]{}
\end{tabular}
\bigskip \\
\begin{tabular}{p{10cm}}
\TextField[name=ov_namnteckning,width=\linewidth,borderstyle=U,borderwidth=0.5]{}\\
Namnteckning\\
\end{tabular}

\end{Form}

\end{document}
