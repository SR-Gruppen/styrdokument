\documentclass[a4paper]{dtekinstruktion}

\setcounter{secnumdepth}{3}
\title{Instruktion: D6}
\date{YYYY-MM-DD}

\begin{document}
\begin{titlepage}
  \thispagestyle{empty} % ingen header/fot på framsidan
  \vspace*{1cm}
  \begin{center}
    \includegraphics[width=250pt]{dteklogo_orange.pdf}\\[3em]
    {\Huge Datateknologsektionen}\\[3em]
    {\Huge \textbf{Instruktion}}\\[1em]
    {\Huge För D6}\\[3em]
    Från DStyret XX/XX till D6 XX
  \end{center}
\end{titlepage}

% Aktivera header/fot (standard enligt dtek.cls)
\makeheadfoot

\tableofcontents

\section{Inledning}
D6 har till uppgift att anordna festliga och sociala aktiviteter för sektionen. Syftet med denna instruktion är att säkerställa att D6:s verksamhet bedrivs på ett organiserat, säkert och professionellt sätt med tydliga arbetsuppgifter för samtliga ledamöter. D6 ska även fungera som en länk mellan sektionen, kårens sociala organ samt DStyret.

\section{Uppdrag och mål}
\begin{itemize}
  \item Anordna minst en större fest eller gasque per läsperiod.
  \item Arrangera pub tillsammans med Delta då det är pubrunda enligt beslut från Chalmers Studentkår.
  \item Skapa en trivsam, social och inkluderande miljö för sektionens medlemmar.
  \item Dokumentera och utvärdera arrangemangen samt lämna in en verksamhetsberättelse till DStyret.
  \item Säkerställa en smidig överlämning till nästkommande mandat.
\end{itemize}

\section{Organisation och sammansättning}
\subsection{Sammansättning}
D6 består av följande roller:
\begin{itemize}
  \item Sexmästare
  \item Sexmästarinna
  \item Kassör
  \item Spritchef
  \item PR-Chef
  \item IT-Riddare
  \item Phixare
  \item Hustomte
\end{itemize}

\section{Arbetsuppgifter och ansvarsområden}
\subsection{Övergripande ledning}
\begin{itemize}
  \item \textbf{Sexmästare} leder D6:s arbete, representerar kommittén gentemot DStyret och säkerställer att alla aktiviteter planeras, genomförs och utvärderas enligt fastställda riktlinjer. Vid mandatperiodens början ska en verksamhetsplan upprättas och vid periodens slut en verksamhetsberättelse lämnas in till DStyret samt sektionens lekmannarevisorer.
  \item \textbf{Kassören} ansvarar för D6:s ekonomi, inklusive budgetering, löpande redovisning och bokföring. Tillsammans med Sexmästare är kassören ekonomiskt ansvarig. Budgeten ska lämnas in till sektionens kassör och lekmannarevisorer.
  \item \textbf{Övriga ledamöter} bistår i det övergripande arbetet och utför sina specifika uppdrag enligt nedan.
\end{itemize}

\section{Rollspecifika åligganden}
WIP


\section{Mötesförfarande och beslutsfattande}
\begin{itemize}
  \item D6 ska sammanträda minst fyra gånger per läsperiod.
  \item Kallelse till möten är inget strikt behov men är ej oönskat.
  \item Protokoll ska föras vid varje möte och distribueras till alla ledamöter inom tre dagar efter mötet. Protokollet ska arkiveras. Alla yttranden ska vara oåterkalliga.
  \item Beslut fattas med enkel majoritet om inte annat anges; ordföranden har utslagsröst vid lika röstetal.
  \item D6:s beslut är giltiga när minst hälften av medlemmarna samt ordföranden är närvarande.
\end{itemize}

\section{Rekrytering}
\begin{itemize}
  \item Rekrytering av nya medlemmar sker enligt DStyrets riktlinjer samt de processer som fastställs vid sektionsmötet.
  \item Valberedningen intervjuar aspiranter till nyckelposter, inklusive Sexmästare/Sexmästarinna och Kassör.
  \item Nya ledamöter ska ges en tydlig bild av D6:s arbetsuppgifter och arbetsbelastning.
\end{itemize}

\section{Överklagande}
\begin{itemize}
  \item D6:s beslut kan i första hand överklagas till DStyret.
  \item DStyret har överbeslutanderätt över D6:s beslut.
  \item Överklaganden ska lämnas in skriftligt enligt fastställda rutiner.
\end{itemize}

\section{Uppföljning och utvärdering}
\begin{itemize}
  \item Efter varje större evenemang ska en utvärdering genomföras med feedback från medlemmar och deltagare.
  \item Identifierade förbättringsområden dokumenteras och tas upp vid planeringen av kommande verksamhetsår.
  \item Regelbundna avstämningar säkerställer att alla uppdrag följs upp och att eventuella avvikelser åtgärdas.
\end{itemize}

\section{Avslutande bestämmelser}
\begin{itemize}
  \item D6:s medlemmar ska uppträda med respekt, professionalitet och samarbetsvilja.
  \item Denna instruktion revideras och uppdateras vid behov inför varje nytt verksamhetsår.
  \item Eventuella ändringar i D6:s verksamhet ska godkännas av DStyret.
\end{itemize}

\end{document}
