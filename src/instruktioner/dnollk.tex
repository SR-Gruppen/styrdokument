\documentclass[a4paper]{dtekinstruktion}

\setcounter{secnumdepth}{3}
\title{Instruktion: DNollK}
\date{Utkast: SR-Gruppen v2026.1-pre3}

\begin{document}
\begin{titlepage}
  \thispagestyle{empty} % no header/footer
  \vspace*{1cm}
  \begin{center}
    \includegraphics[width=250pt]{dteklogo_orange.pdf}\\[3em]
    {\Huge Datateknologsektionen}\\[3em]
    {\Huge \textbf{Instruktion}}\\[1em]
    {\Huge För DNollK}\\[3em]
    Från DStyret XX/XX till DNollK XX
  \end{center}
\end{titlepage}

% After front page, we activate fancyhdr
\makeheadfoot

\tableofcontents

\section{Inledning}
DNollK har till uppgift att sköta mottagningen samt att värna om sektionens phaddergrupper under årets gång. Syftet med denna instruktion är att säkerställa att verksamheten bedrivs på ett organiserat, professionellt och trivsam sätt med tydliga arbetsuppgifter för samtliga ledamöter. DNollK ska även fungera som en länk mellan kårens mottagningsorgan, Chalmers institution för Data- och informationsteknik samt DStyret.

\section{Uppdrag och mål}
\begin{itemize}
  \item Planera och genomföra mottagningsaktiviteter i samarbete med kårens mottagningskommitté, Chalmers institution för Data- och informationsteknik samt sektionens övriga kommittéer, utskott och föreningar.
  \item Säkerställa att mottagningen genomförs enligt sektionens och Chalmers studentkårs policies och riktlinjer.
  \item Värna om och stödja phaddergrupperna under hela mandatperioden.
  \item Dokumentera och utvärdera mottagningsverksamheten samt lämna in en verksamhetsberättelse till DStyret och sektionens revisorer.
  \item Överlämna ansvaret till nästkommande mandat.
\end{itemize}

\section{Organisation och sammansättning}
\subsection{Sammansättning}
DNollK består av följande roller:
\begin{itemize}
  \item Ordförande
  \item Kassör
  \item PR-chef
  \item Eventchef
  \item Modulchef
  \item Phadderchef
  \item Nollupdrags-/mottagningskamps ansvarig (postnamnet kan variera från år till år)
  \item Sponschef
\end{itemize}

\section{Arbetsuppgifter och ansvarsområden}
\subsection{Övergripande ledning}
\begin{itemize}
  \item \textbf{Ordföranden} leder DNollK:s arbete, representerar kommittén i DStyret samt samordnar planering, genomförande och utvärdering av mottagningsaktiviteter. Ordföranden ansvarar även för att upprätta en verksamhetsplan vid mandatperiodens början samt en verksamhetsberättelse vid mandatperiodens slut enligt tillhandahållen mall. Dessa dokument ska sedan lämnas in till DStyret och sektionens revisorer.
  \item \textbf{Kassören} ansvarar för DNollK:s ekonomi, budgetuppföljning samt löpande redovisning enligt gällande riktlinjer. Tillsammans med ordföranden är kassören ekonomiskt ansvarig. Budgeten ska lämnas in till sektionens kassör och revisorer.
  \item \textbf{Övriga ledamöter} bistår i det övergripande arbetet och hjälper till med att driva kommitténs olika uppdrag.
\end{itemize}

\section{Rollspecifika åligganden}
\subsection{PR-chef}
\begin{itemize}
  \item Ansvara för att planera och genomföra kommunikationsinsatser inför och under mottagningen.
  \item Uppdatera DNollK:s digitala kanaler, exempelvis hemsida och sociala medier.
  \item Informera medlemmar och potentiella deltagare om aktiviteter och nyheter.
  \item Samarbeta med övriga roller för att säkerställa en enhetlig och tydlig kommunikation.
\end{itemize}

\subsection{Eventchef}
\begin{itemize}
  \item Planera, organisera och genomföra mottagningsaktiviteter med särskilt fokus på event och sociala inslag.
  \item Ansvara för logistik, lokalbokning och övergripande evenemangsplanering.
  \item Utarbeta en tidsplan och budget för varje evenemang i nära samarbete med kassören.
\end{itemize}

\subsection{Modulchef}
\begin{itemize}
  \item Ansvara för att skapa \emph{nollmodulen} – en broschyr som skickas ut under sommaren till alla programstartande studenter. I broschyren introduceras sektionen, studier vid Chalmers samt information om vad som väntar under nollans fyra första mottagningsveckor.
  \item Samverka med övriga roller och organ i sektionen för att de ska bli representerade i nollmodulen.
\end{itemize}

\subsection{Phadderchef}
\begin{itemize}
  \item Organisera och övervaka phaddergrupperna under mottagningen.
  \item Ansvara för att phadderkontrakt upprättas och följs upp.
  \item Erbjuda stöd och löpande uppföljning för att säkerställa att de nyantagna får en bra introduktion.
\end{itemize}

\subsection{Nollupdrags-/mottagningskamps ansvarig}
\begin{itemize}
  \item Ansvara för att skapa \emph{nolluppdrag} – uppdrag som nollan kan tilldelas i början av mottagningen. Dessa uppdrag, som syftar till att främja samhörighet, ska först skapas av den ansvarige, därefter godkännas av kårens mottagningskommitté (MK) innan de tilldelas nollan.
  \item Ansvara för att i slutet av mottagningen genomföra en \emph{nolluppdragsredovisning} där uppdragen utvärderas.
  \item Ansvara för att arrangera D-sektionens bidrag till \emph{mottagningskampen}.
\end{itemize}

\subsection{Sponschef}
\begin{itemize}
  \item Ansvara för att identifiera och kontakta potentiella sponsorer.
  \item Förhandla och upprätta sponsoravtal i samarbete med ordföranden kassören och DAG.
  \item Följa upp att sponsorernas åtaganden efterlevs under mottagningsperioden.
  \item All kontakt med sponsorer ska ske i samråd med DAG.
  \item Samordna med övriga roller för att integrera sponsoraktiviteter i mottagningen.
\end{itemize}

\section{Mötesförfarande och beslutsfattande}
\begin{itemize}
  \item DNollK ska sammanträda minst fyra gånger per läsperiod.
  \item Kallelse till möten är inget strikt behov men är ej oönskat.
  \item Protokoll ska föras vid varje möte och distribueras till alla ledamöter inom tre dagar efter mötet. Protokollet ska arkiveras. Alla yttranden ska vara oåterkalliga.
  \item Beslut fattas med enkel majoritet om inte annat anges; ordföranden har utslagsröst vid lika röstetal.
  \item DNollK:s beslut är giltiga när minst hälften av medlemmarna samt ordföranden är närvarande.
\end{itemize}

\section{Rekrytering}
\begin{itemize}
  \item För rekrytering av nya medlemmar i DNollK ansvarar kommittén enligt DStyrets riktlinjer.
  \item Aspiranter till samtliga poster ska intervjuas av valberedningen.
  \item Aspiranter ska ges en tydlig bild av arbetsuppgifter och arbetsbelastning inom DNollK.
  \item Vid förfrågan ska skriftlig motivering ges för nominering eller icke-nominering. Eventuella invändningar hänvisas till DStyret.
\end{itemize}

\section{Överklagande}
\begin{itemize}
  \item DNollK:s beslut kan i första hand överklagas till DStyret.
  \item DStyret har överbeslutanderätt över DNollK:s beslut.
  \item Överklaganden ska lämnas in skriftligt.
\end{itemize}

\section{Uppföljning och utvärdering}
\begin{itemize}
  \item Efter avslutad mottagning ska en utvärdering genomföras med feedback från både medlemmar och deltagare.
  \item Identifierade förbättringsområden dokumenteras och tas upp vid planeringen av kommande verksamhetsår.
  \item Regelbundna avstämningar säkerställer att alla uppdrag följs upp och att eventuella avvikelser hanteras snabbt.
\end{itemize}

\section{Avslutande bestämmelser}
\begin{itemize}
  \item DNollK ansvarar för att samtliga aktiviteter genomförs med respekt, professionalitet och god ton.
  \item Denna instruktion revideras och uppdateras vid behov inför varje nytt verksamhetsår.
  \item Eventuella ändringar i DNollK:s verksamhet ska godkännas av DStyret.
\end{itemize}

\end{document}
