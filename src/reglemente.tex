\documentclass[a4paper]{dteklag}

\setcounter{secnumdepth}{5}
\title{Reglemente}
\date{Utkast: SR-Gruppen Ver 2}

%%%%%%%%%%%%%%%%%%%%%%%%%%%%%%%%
% PRISBASBELOPP
% sätt till nuvarande prisbasbelopp för att annotera med prisbasbelopp i spalten till höger.
% OBS detta ska vara avstängt för det faktiska reglementesdokumentet och när ändringar ska godkännas på sektionsmöten
%\newcommand{\nuvarandeprisbasbelopp}{47600}

\newcommand{\prisbasbelopp}[1]{
    #1 prisbasbelopp
    \ifdef{\nuvarandeprisbasbelopp}{
        \marginpar{
            \small{ \textbf{\roundandprint{\xintiexpr#1 * \nuvarandeprisbasbelopp\relax}kr}}
        }
    }{}
}

\begin{document}
\begin{titlepage}
  \thispagestyle{empty} % no header/footer
  \vspace*{1cm}
  \begin{center}
    \includegraphics[width=250pt]{dteklogo_orange.pdf}\\[3em]
    {\Huge Datateknologsektionen}\\[3em]
    {\Huge \textbf{Reglemente}}\\[1em]
    Utkast: SR-Gruppen Ver 2
  \end{center}
\end{titlepage}

% After front page, we activate fancyhdr
\makeheadfoot

\setcounter{tocdepth}{2}
\tableofcontents

\section{Allmänt}
\begin{center}
  [Inga regleringar utöver Sektionens stadga]
\end{center}
\section{Medlemmar}
\begin{center}
  [Inga regleringar utöver Sektionens stadga]
\end{center}
\section{Organisation}
\para För ledamot i valberedningen ska arvode bestämmas utifrån antalet intervjuer ledamoten närvarat på.
\para För rättigheter får inte pengar spenderas på alkohol om inte annat anges.
\para Rättigheter får inte överskrida kommitténs ekonomiska resurser.
\para Varje organ har inte rätt att belasta sin ekonomi med mer än:
\begin{itemize}
    \item[-] \prisbasbelopp{0.004} för teambuildingaktiviteter per person.
    \item[-] \prisbasbelopp{0.0020} (avrundat till närmaste femkrona) för överlämning per person som deltar från avgående samt påstigande år.
    \item[-] \prisbasbelopp{0.005}  per post för aspning och aspplagg sammanlagt.
\end{itemize}
\para Vid arrangemang för sektionen kan representation eller möteskostnader beviljas. Dessa belastar kommittéens ekonomi. Hur mycket som godkänns baseras på arrangemangets längd där beloppet avrundas till närmaste femkrona:
\begin{itemize}
    \item 1 till 4 timmar: \prisbasbelopp{0.00075} per person.
    \item 4 till 8 timmar: \prisbasbelopp{0.0015} per person.
    \item Mer än 8 timmar: \prisbasbelopp{0.00225} per person.
\end{itemize}
\para Vid redovisning av representation eller möteskostnader ska deltagare samt tiden de deltagit finnas med för att belopp ska kunna godkännas.
\para[Representationskläder] Varje kommitté har rätt att köpa representationskläder till varje medlem 
för att synliggöra sig själva på sektionen och campus. Representationskläder 
kan ses som, overaller, arbetsbyxor, hoodies, t-shirts eller liknande.
Kostnaden för representationskläder får inte överstiga 0.01 prisbasbelopp 
per person i kommittén.

\para
Varje kommitté har rätt att skicka in en förfrågan för att använda upp
till 0.022 prisbasbelopp per person i kommitéen. För att få lov att köpa 
in representationskläder för denna summa ska en kommitté få godkännande 
av styrelsen innan inköp. Detta genom att kommittén skickar förslag på 
kläder och en uppskattad kostnad som styrelsen baserar sitt beslut på.

\para
En ansökan om representationskläder ska innehålla:
\begin{itemize}
  \item Det som man vill köpa in och vad som tidigare har köpts in.
  \item När och i vilka sammanhang representationskläder kommer användas.
  \item Varför gruppen vill ha representationskläder.
  \item En ungefärlig kostnadsberäkning.
\end{itemize}
\para[Teambuilding] För att få använda hela det budgeterade beloppet för teambuilding måste 50\% av beloppet användas inom 8 månader från mandatperiodens början. Annars får inte mer än det som spenderats alternativt upp till 25\% användas.
\para Upp till en tredjedel av beloppet för teambuildingaktiviteter får användas till alkohol per förtroendepost i organ.
\para [Aspning] Varje kommitée har rätt att skicka in en förfrågan för att använda upp till
0.017 prisbasbelopp per person i kommitéen för aspning och aspplagg sammanlagt. För att
få lov att spendera denna summa på aspning ska en kommitté få godkännande av
styrelsen innan inköp. Detta genom att kommittén skickar in plan för aspningen samt
en kostnadskalkyl som styrelsen baserar sitt beslut på.
\para[Sponsring] Om kommittéer vill söka sponsring ska detta göras i samråd med datas arbetsmarknadsgrupp, DAG.
\section{Sektionsmötet}
\para Medlem eller kommittéer kan också inkomma med önskemål om extra ekonomiska medel (äskning) till sin verksamhet eller arrangemang genom motion till sektionsmötet. Finansieringen av motionen ska göras med prioritet på medel från de fonder som styrelsen förvaltar.
\section{Valberedningen och personval}
\begin{center}
  [Inga regleringar utöver Sektionens stadga]
\end{center}
\section{Sektionsstyrelsen}
\para Det åligger sektionsordförande att:
\begin{itemize}
  \item tillse att sektionens beslut verkställs
  \item föra sektionens talan då något annat ej stadgats eller beslutats
  \item teckna sektionens firma
  \item leda och övervaka arbetet inom sektionsstyrelsen
  \item representera sektionen på kårledningsutskottet
  \item till varje sektionsmöte kunna redogöra om sektionens verksamhet
  \item se till att ordförande i varje kommitté har tillgång till och kunskap omsektionens stadgar, reglemente och policies
  \item Se till att ordförande i varje kommitté skriver en verksamhetsrapport inför varje brytpunkt.
\end{itemize}
\para Det åligger sektionens vice ordförande att:
\begin{itemize}
  \item biträda ordföranden i dennes värv
  \item i ordförandens frånvaro överta dennes åligganden
  \item kontrollera så att arbetet sker i enighet med sektionens bestämmelser
  \item representera sektionen på kårens nöjeslivsutskott
\end{itemize}
\para Det åligger styrelsens kassör att:
\begin{itemize}
  \item teckna sektionens firma
  \item se till att kassör i varje kommitté har tillgång till och kunskap om sektionens stadgar, reglemente och policies
  \item Fortlöpande kontrollera sektionens räkenskaper och bokföring
  \item representera sektionen på kårens sektionsekonomiforum
  \item i samråd med styrelsen upprätta budgetförslag till första ordinarie höstmötet
  \item till varje sektionsmöte kunna redogöra för sektionens ekonomiska ställning
  \item utbilda nya förtroendevalda i hur sektionens bokförings och redovisningssystem skall användas
\end{itemize}
\para Det åligger styrelsens sekreterare att:
\begin{itemize}
  \item föra protokoll vid styrelsens möten och tillse att protokoll från såväl styrelse- som sektionsmöten anslås
  \item tillse att material som inkommer till sektionen anslås eller på annat sätt förmedlas till berörda parter
  \item tillse att sektionens styrdokument hålls uppdaterade i enlighet med sektionsmötes- och styrelsebeslut
\end{itemize}
\para Det åligger sektionens SAMO att:
\begin{itemize}
  \item verka för studenternas trivsel på sektionen.
  \item bistå studenter i frågor kring fysisk och psykisk hälsa
  \item föra sektionens talan i frågor kring psykosocial och fysisk studie- och arbetsmiljö.
  \item representera sektionen på kårens sociala utskott
\end{itemize}
\para Det åligger styrelsens övriga medlemmar att:
\begin{itemize}
  \item bistå styrelsen med information
  \item aktivt deltaga i beslutsprocessen
  \item redogöra för sin egen eller sin kommittés löpande verksamhet vid styrelsens möten
\end{itemize}
\para [Ekonomi] Styret har rätt att utföra kortfristiga lån till kommittéer.
\para Kortfristiga lån till kommittéer ska betalas tillbaka senast i slutet av lånetagares mandatperiod.
\para Medlem eller kommittéer som önskar extra ekonomiska medel till sin verksamhet eller arrangemang skall inkomma med önskemål och skäl till styret.
\stycke Styret kan bevilja extra medel om beloppet faller inom ramarna för de olika fonderna som styrelsen förvaltar.
\stycke Om äskningen inte fallar inom dessa ramar kan den ändå beviljas förutsatt att beloppet understiger \prisbasbelopp{0.25} och inte överstiger sektionens budget.
\para När styrelsen beslutar huruvida ett förslag godkänns ska följande tas hänsyn till:
\begin{itemize}
    \item Att priset inte är onödigt dyrt (jämfört med liknande kläder för andra kommittéer).
    \item Att kläderna har en anknytning till verksamheten.
    \item Att kommittén kommer använda kläderna tillräckligt mycket för att motivera inköp.
\end{itemize}

\section{Ekonomi}
\para Det prisbasbelopp som skall användas under hela verksamhetsåret är det prisbasbelopp som är aktuellt vid verksamhetsårets början.
\para Delar av prisbasbelopp skall avrundas till närmaste hundratal.
\para De fonder som beskrivs i denna del syftar till sparande av medel för framtida användande enligt respektive fonds syfte, antingen planerat eller spontant. Ett uttag ur en fond skall inte belasta Sektionsstyrelsens godkända budget, däremot skall inköp och investeringar vara föremål för redovisning i balans– respektive resultaträkning. Detta gäller om inget annat specificeras i fonden.
\para Sektionsmötet äger alltid rätt att göra uttag ur fonder i enhet med fondens syfte. Sektionsmötet äger rätt att vid 2/3 majoritet göra uttag ur fonder i strid mot fondens syfte, ett välmotiverat behov till uttaget skall protokollföras vid ett sådant beslut. Alla andra uttagsformer måste specificeras i det ekonomiska reglementet.
\para Alla uttag ur fonder ska redovisas på sektionsmöte i styrelsens verksamhetsrapport.
\para Sektionens fonder förvaltas av sektionsstyrelsen om inte annat regleras.
\para[Kapitalfonden] Syftet med kapitalfonden är att avlasta sektionens respektive de olika sektionskommittéernas ekonomi från stora investeringar. 
\para Kapitalfonden skall användas till saker som har ett bestående värde, samt är till gagn för sektionens medlemmar direkt eller indirekt.
\para Kapitalfonden skall ej användas till driftbidrag eller stöd för förgänglig verksamhet.
\para Kapitalfonden skall ej användas till verksamhet som lokalfond är ämnad för.
\para Till kapitalfonden ska följande avsättas:
\begin{itemize}
\item En av styrelsen budgeterad summa som godkänts av sektionsmötet
\item All avkastning ifrån kapitalfonden under verksamhetsåret.
\end{itemize}
\para Sektionsstyrelsen har rätt att bevilja uttag ur fonden på belopp upp till totalt \prisbasbelopp{0.25} per tillfälle. Uttag av belopp överstigande \prisbasbelopp{0.25} skall godkännas av sektionsmöte innan medel utbetalas.
\para Sektionsstyrelsen äger inte rätt att ta ut mer än 50\% av fondens totala värde per tillfälle.
\para[Lokalfonden] Syftet med lokalfonden är att säkra medel för underhåll och reparationer av sektionens lokaler, samt för inköp av inventarier.
\para Lokalfonden skall användas till större reparationer och ommålningar av sektionslokalerna, samt möbler till trivselytor där teknologen i gemen har tillträde.
\para Till lokalfonden ska följande avsättas:
\begin{itemize}
\item Minst Tio (10)\% av under verksamhetsåret influtna sektionsavgifter tillförs lokalfonden.
\item En av styrelsen budgeterad summa som godkänts av sektionsmötet
\item All avkastning ifrån lokalfonden under verksamhetsåret.
\end{itemize}
\para Sektionsstyrelsen disponerar sjuttiofem (75) \% av årets tillförda kapital, enligt \S\ref{sec:lokalfond_avsattning}, för basdrift av sektionslokalerna.
\para Sektionsstyrelsen har rätt att besluta om ytterliggare uttag, dock skall detta redovisas inför nästkommande sektionsmöte. Vid detta sektionsmöte skall i sådana fall även genomförda eller planerade inköp, reparationer och underhåll redovisas.
\para[Bilfonden] Syftet med bilfonden är att bygga upp en buffert för bilinköp och minska belastningen av sektionens ekonomi vid omfattande skador.
\para Till bilfonden ska följande avsättas:
\begin{itemize}
\item En av styrelsens budgeterad summa som godkänts av sektionsmötet.
\item All avkastning ifrån bilfonden under verksamhetsåret.
\end{itemize}
\para Sektionsstyrelsen äger rätt att bevilja uttag för omfattande reparation av sektionensbil.
\para Sektionsstyrelsen har rätt att köpa in en ny bil efter att ha fått en proposition till sektionsmötet godkänd med en specifikation på kostnad och krav för bilen.
\para[Idéfonden] Syftet med idéfonden är att möjliggöra för datateknologer som önskar medel till arrangemang eller inventarier vars effekt är till gagn för alla sektionens medlemmar.
\stycke Kommittéer som önskar medel för sådana ändamål är inte föremål för idéfonden.
\para Till idéfonden ska följande avsättas:
\begin{itemize}
\item Tio (10) \% av under verksamhetsåret influtna sektionsavgifter tillförs idéfonden.
\item En av styrelsen budgeterad summa som godkänts av sektionsmötet.
\item All avkastning ifrån idéfonden under verksamhetsåret.
\end{itemize}
\para Sektionsstyrelsen har rätt att, på förslag av medlem i teknologsektionen, bevilja uttag på upp till 0,25 prisbasbelopp som går i hand med idéfondens syfte.
\para Sektionsmötet har rätt att, på förslag av medlem i teknologsektionen, bevilja uttagsom går i hand med idéfondens syfte.
\para[Husfonden] Syftet med husfonden är att möjliggöra framtida fastighetsinvestering eller lokalför-värvning för sektionensmedlemmars bästa.
\para Till husfonden ska följande avsättas:
\begin{itemize}
  \item En av styrelsen budgeterad summa som godkänts av sektionsmötet.
  \item All avkastning ifrån husfonden under verksamhetsåret.
\end{itemize}
\para Sektionsstyrelsen har rätt att göra uttag från husfonden som uppfyller syftet efter att ha fått en proposition till sektionsmötet godkänd som inkluderar en kostnadskalkyl och annan relevant information om investeringen.
\para[Jubileumsfonden] Syftet med Jubileumsfonden är att möjliggöra firande av sektionens framtida jubileum, genom att möjliggöra större och dyrare evenemang.
\para Till jubileumfonden ska följande avsättas:
\begin{itemize}
  \item En av styrelsen budgeterad summa som godkänts av sektionsmötet.
  \item All avkastning ifrån Jubileumsfonden under verksamhetsåret.
\end{itemize}
\para Sektionsstyrelsen har rätt att göra uttag under jubileumsår.

\section{Kommittéer}
\para Teknologsektionens kommittéer är:
\begin{itemize}
  \item DRust
  \item DAG
  \item Delta
  \item D6
  \item DNollK
  \item DMNollK
\end{itemize}
\para Ordförande i kommitté är skyldig att
\begin{itemize}
  \item kontinuerligt meddela kommitténs ekonomiska status till styret
  \item tillsammans med kommitténs kassör ansvara för att kommittén förvaltar sina tillgångar i enlighet med sektionens stadgar, reglementen och beslut.
\end{itemize}
\para Kassör i kommitté är skyldig att
\begin{itemize}
  \item föra kassabok av sådan typ som godkänts av sektionens revisorer
  \item tillsammans med kommitténs ordförande ansvara för att kommittén förvaltar sina tillgångar i enlighet med sektionens stadgar, reglementen och beslut.
  \item tillsammans med kommitténs ordförande skriva ett bindandeavtal med Styret angående skuldfrågan vid felaktigt förd bokföring
  \item på varje sektionsmöte redovisa kommitténs ekonomiska situation
  \item arkivera kommitténs bokföring, på en plats anvisad av sektionsstyrelsen, så lång tid som föreskrivs för den organisationsform som sektionen är.
  \item lägga en budget
\end{itemize}
\para Kommitté ska i början av året lägga en budget för hur dess pengar ska spenderas.
\stycke Styrelsens kassör ska delge en mall för budget för att underlätta kommittés arbete.
\stycke I budgeten ska det tydligt framgå hur pengarna ska spenderas med en uppskattning för vilken månad utgifter och inkomster hamnar i.
\para Varje kommittés budgetförslag ska godkännas av styrelsen, som rapporterar vidare relevant information till sektionsmötet.
\para Kommitté skall vid mandatperiodens början inte ha några tillgångar.
\para Kommitté har rätt att låna \prisbasbelopp{0.25} från sektionsstyrelsen i form av startlån.
\para Kommittés tillgångar och skulder tillfaller sektionsstyrelsen vid mandatperiodens slut.
\para[DRust] DRust har i uppdrag att ansvara för sektionens lokaler, fordon och allt som har samröre med detta.
\para DRust består av ordförande, kassör och 5 övriga ledamöter.
\para[DAG] DAG har i uppdrag att verka för samverkan mellan sektionen och arbetsmarknaden.
\para DAG består av ordförande, kassör och 5 övriga ledamöter.
\para DAG har rätt låna ytterligare \prisbasbelopp{1} i form av utökat startlån.
\para[Delta] Delta har i uppdrag att verka för sektionsfrämjande aktiviteter.
\para Delta består av ordförande, kassör och 4 övriga ledamöter.
\para[D6] D6 har i uppdrag att arrangera fester.
\para Ordförande i D6 ska representera sektionen på kårens gasqueforum.
\para D6 består av ordförande, kassör och 6 övriga ledamöter.
\para[DNollK] DNollK har i uppdrag att sköta mottagningen och att värna om sektionens phaddergrupper under årets gång.
\para Ordförande i DNollK ska representera sektionen på kårens mottagningsforum.
\para DNollK består av ordförande, kassör och 6 övriga ledamöter.
\para Mandatperioden för DNollK är 1:a januari – 31:a december.
\para[DMNollK] DMNollK har i uppdrag att sköta mastermottagningen.
\para DMNollK består av ordförande, kassör och 5 övriga ledamöter.
\para Mandatperioden för DMNollK är 1:a januari – 31:a december.
\section{Utskott}
\para Teknologsektionens utskott är:
\begin{itemize}
  \item DLirium
  \item dHack
  \item Ståthållarämbetet
  \item JämställD
  \item DFoto
  \item DKock
\end{itemize}
\para För utskott ansvarar styrelsens kassör för samtliga av punkterna i \S\ref{sec:kommittee_kassor} med undantag för att lägga budget vilket görs i samråd med kommittéens ordförande
\para[DLirium] Ska roa och kritiskt granska sektionen, kåren och Chalmers.
\para DLirium består av ordförande och 3 redaktörer.
\para[dHack] Ska driva teknologsektionens IT-tjänster och datorsystem, samt främja hackerandan på sektionen.
\para dHack består av en ordförande och 3 övriga ledamöter.
\para[Ståthållarämbetet] Se efter d08gong med tillbehör samt sektionens fana, representera sektionen på traditionsenliga arrangemang, arrangera sektionens årliga firande av Hackes födelsedag. och andra högtidliga arrangemang samt hålla alumnkontakt.
\para Ståthållarämbetet består av fanbärare tillika ordförande, ceremonimästare och 4 vapendragare.
\para[JämställD] JämställD ska främja jämlikhet och välmående på datateknologsektionen.
\para JämställD består av ordförande, Teknologsektionens SAMO och 4 övriga ledamöter.
\para[DFoto] DFoto har i uppdrag att i bild dokumentera teknologsektionens verksamhet.
\para DFoto består av ordförande och 5 övriga ledamöter
\para[DKock] DKock ska främja datateknologers matlagningsintresse samt vid behov hjälpa övriga kommittéer med matlagning.
\para DKock består av ordförande och 5 övriga ledamöter
\para DKock har rätt att låna \prisbasbelopp{0.25} från sektionsstyrelsen i form av startlån.
\section{Intresseföreningar}
Teknologsektionens intresseföreningar är:
\begin{itemize}
  \item Datas tyska mousserande vin
  \item DAF
  \item Datas Ludologer
  \item plantaD
  \item bakaD
  \item Datas Sommelierliga
  \item Datas tyska mousserande vin
  \item iDrott
\end{itemize}
\section{Inspektor}
\begin{center}
  [Inga regleringar utöver Teknologsektionens stadga]
\end{center}
\section{Teknologsektionens identitet}
\begin{center}
  [Inga regleringar utöver Teknologsektionens stadga]
\end{center}
\section{Protokoll och anslagning}
\begin{center}
  [Inga regleringar utöver Teknologsektionens stadga]
\end{center}
\section{Revision och ansvarsfrihet}
\begin{center}
  [Inga regleringar utöver Teknologsektionens stadga]
\end{center}
\section{Teknologsektionens upplösning}
\begin{center}
  [Inga regleringar utöver Teknologsektionens stadga]
\end{center}
\section{Ändrings- och tolkningsfrågor}
\begin{center}
  [Inga regleringar utöver Teknologsektionens stadga]
\end{center}
\section{Hedersbetygelser}
\begin{center}
  [Inga regleringar utöver Teknologsektionens stadga]
\end{center}

\end{document}
