%   stadga.tex
%   KOMPILERAS MED xelatex!
\documentclass[a4paper]{dtek}

\setcounter{secnumdepth}{5}
\title{Stadga}
\date{Utkast: SR-Gruppen Ver 2}

% If you need additional packages just for stadga.tex, put them here:
%\usepackage{...}

\begin{document}
\begin{titlepage}
  \thispagestyle{empty} % no header/footer
  \vspace*{\fill}
  \begin{center}
    \includegraphics[width=250pt]{dteklogo_orange.pdf}\\[3em]
    {\Huge Datateknologsektionen}\\[3em]
    {\Huge \textbf{Stadga}}\\[1em]
    Utkast: SR-Gruppen Ver 2\\[20em]
  \end{center}
  \vspace*{\fill}
\end{titlepage}

% After front page, we activate fancyhdr
\makeheadfoot

\setcounter{tocdepth}{2}
\tableofcontents

%Allmänt%%%%%%%%%%%%%%%%%%%%%%%%%%%%%%%%%%%%%%%%%%%%%%%%%%%%%%%%%
\section{Allmänt}
\para[Ändamål] Datateknologsektionen vid Chalmers tekniska högskola, härefter kallad \textit{sektionen}, är en ideell förening av studerande vid civilingenjörsprogrammet Datateknik vid Chalmers samt de masterprogram som är ackrediterade till detta. Sektionen utgör en teknologsektion under Chalmers Studentkår. 
\para Sektionen har till uppgift att främja sammanhållningen bland sina medlemmar och att tillvarata deras gemensamma intressen i utbildningsfrågor och studiesociala frågor. Sektionen verkar utan vinstintresse.
\para Sektionen och dess medlemmar ska följa Chalmers Studentkårs stadga, reglementen och beslut. Beslut eller stadgeändring inom sektionen som strider mot kårens stadgar, reglementen, policys eller beslut får upphävas av Chalmers Studentkårs fullmäktige. Sektionen ska hålla sina styrdokument uppdaterade i enlighet med kårens riktlinjer. 

\para Sektionen är fackligt, partipolitiskt och religiöst obunden och vilar på demokratisk grund.
\para Sektionen har sitt säte i Göteborg.
\para[Verksamhetsår]
sektionens verksamhetsår löper från och med den första maj.
\para Sektionens verksamhetsår löper från och med 1 maj till och med 30 april. Mandatperioden för sektionens förtroendevalda är 1 maj – 30 april.
\para[Räkenskapsår]
Sektionens räkenskapsår sammanfaller med verksamhetsåret (1 maj – 30 april), om inte annat beslutas av sektionen.

%Medlemmar%%%%%%%%%%%%%%%%%%%%%%%%%%%%%%%%%%%%%%%%%%%%%%%%%%%%%%%%%
\section{Medlemmar}
\para Medlem har rätt att ta del av mötesprotokoll och sektionens övriga handlingar.
\para Medlem är skyldig att rätta sig efter sektionens bestämmelser.
\para[Datateknologer] Ordinarie medlem i sektionen, härmed benämnd datateknolog, är den som är inskriven vid utbildningsprogrammet Datateknik civilingenjör vid Chalmers eller ett av dess associerade masterprogram. 
\para sektionen kan ha hedersmedlemmar och stödmedlemmar.
\stycke Dessa medlemmar är inte datateknologer.
\para Datateknolog ska ha erlagt sektionsavgift.
\para[Hedersmedlem] Till hedersmedlem kan kallas person som synnerligen främjat sektionens intressen och strävande.
\para Förslag till hedersmedlem lämnas i skrivelse till sektionsstyrelsen senast sju läsdagar innan sektionsmötet undertecknad av minst 25 av sektionens medlemmar.
\para Beslut om kallande av hedersmedlem fattas vid nästkommande sektionsmöte och är enbart giltigt om det antas med två tredjedelar av antalet röster.
\para Antager kallad person en kallelse till hedersmedlem är denne officiellt hedersmedlem.
\para[Stödmedlem] Stödmedlem är den person som tidigare varit datateknolog på sektionen samt har erlagt en administrationsavgift till sektionen.
\para[Avgifter] Sektionens medlemsavgift fastställs av Chalmers Studentkår.
\para Stödmedlem skall erlägga en med styrelsen beslutad administrationsavgift.
%Organisation%%%%%%%%%%%%%%%%%%%%%%%%%%%%%%%%%%%%%%%%%%%%%%%%%%%%%%%%%
\section{Organisation}
\para[Sektionens organ] Sektionens verksamhet utövas i enlighet med denna stadga, tillhörande reglementen och beslut genom följande organ:
\begin{itemize}\setlength\itemsep{0em}
\item \textit{Sektionsmötet} – sektionens högsta beslutande organ (kap.~4),
\item \textit{Sektionsstyrelsen} – sektionens verkställande organ (kap.~6),
\item \textit{Datatekniks Nämnd för Studier (DNS)} – utbildningsorgan för studiebevakning (kap.~7),
\item \textit{Sektionskommittéer} – kommittéer för sektionens verksamhet (kap.~8),
\item \textit{Sektionsutskott} – grupper med permanenta uppdrag utan egen kassör (kap.~9),
\item \textit{Intresseföreningar} – fristående föreningar som ägs av sektionen (kap.~10),
\item \textit{Valberedningen} – organ för nominering av förtroendevalda (kap.~5),
\item \textit{Revisorer} – organ för granskning av verksamhet och ekonomi (kap.~12),
\item \textit{Inspektor} – rådgivande stöd och överinseende (kap.~15).
\end{itemize}
\stycke Sektionen arrangerar även en arbetsmarknadsmässa (DatE-IT) i samarbete med Elektrosektionen Chalmers Studentkår (857202-2013) och sektionen Informationsteknik (857209-9524). Denna verksamhet regleras av DatE-ITs styrdokument. 
\para[Ansvarsförhållanden] Sektionsmötet har till sitt förfogande valberedning, revisorer, DNS, kommittéer, utskott, intresseföreningar och sektionsstyrelsen.
\para Sektionsstyrelsen har till sitt förfogande kommittéer, utskott och intresseföreningar.
\para[Ansvarspost] Ansvarspost är ledamot i sektionsstyrelsen samt ordförande och kassör i sektionens organ.
\para Medlem får ej inneha mer än en ansvarspost samtidigt på sektionen om inte
sektionsmötet finner det lämpligt med kvalificerad (2⁄3) majoritet.
\para Innehavaren av ansvarspost skall vara myndig.
\para[Förtroendepost] Förtroendepost innehas av en person som av sektionsmötet är invald till sektionens organ, inspektor eller revisor.
\para Endast datateknolog är valbar till förtroendepost inom sektionen. 
\stycke Undantag är kandidater till revisor, valberedning och inspektor som även utomstående är valbara till.
\para[Misstroendeförklaring] Misstroendeförklaring kan riktas mot enskild förtroendevald eller ett sektionsorgan, exklusive sektionsföreningar, i sin helhet.
\para Rätt att rikta misstroendeförklaring tillfaller:
\begin{itemize}
  \item två sektionsrevisorer.
  \item två sektionsstyrelseledamöter.
  \item tre ordföranderådsmedlemmar.
  \item 25 medlemmar.
\end{itemize}
\para Misstroendeförklaring resulterar i misstroendevotum inför sektionsmötet. Misstroendeförklaring behandlas som motion.
\para Misstroendeförklaring bifalles då minst 35 medlemmar och en kvalificerad (2⁄3) majoritet av de röstberättigade vid sektionsmöte är om beslutet ense.
\para Bifallen misstroendeförklaring mot enskild förtroendevald resulterar i att den förtroendevalda entledigas från sitt uppdrag.
\para Då sektionsstyrelsen har entledigats i sin helhet skall en interimstyrelse tillsättas. Interimstyrelsen utfärdar kallelse till extra sektionsmöte där ny ordinarie styrelse skall väljas. Detta sektionsmöte skall hållas inom 10 läsdagar. Interimstyrelsen övertar sektionsstyrelsens befogenheter och åtaganden, men får endast handha löpande ärenden.
\para Då studienämnden har entledigats i sin helhet skall en interim studienämnd tillsättas. Sektionsstyrelsen utfärdar kallelse till extra sektionsmöte där ny ordinarie studienämnd skall väljas. Detta sektionsmöte skall hållas inom 10 läsdagar. Den interima studienämnden övertar studienämndens befogenheter och åtaganden, men får endast handha löpande ärenden.
\para Då revisor entledigas så att det totala antalet revisorer understiger det stadgeenliga skall interima revisorer tillsättas. På nästkommande sektionsmöte skall erforderligt antal revisorer tillsättas.
\para[Avsägelse] Avsägelse inlämnas skriftligen till sektionsstyrelsen senast tre dagar före sektionsmötet för fastställande.
\para[Datatekniks Nämnd för Studier] Datatekniks Nämnd för Studier, härmed benämnd studienämnden eller DNS....
\para[Ordföranderåd] Kommer snart
\para[Arvode] Sektionsordförande, sektionskassör, ledamot i valberedningen och revisor har rätt att begära arvode för deras arbete.
\para Begärd arvode betalas ut efter beviljad ansvarsfrihet.

%Sektionsmötet%%%%%%%%%%%%%%%%%%%%%%%%%%%%%%%%%%%%%%%%%%%%%%%%%%%%%%%%%
\section{Sektionsmötet}
\para[Sammanträden] Det skall hållas fyra ordinarie sektionsmöten, ett per läsperiod. Utöver detta kan extra sektionsmöten hållas.
\para Sektionsmötet sammanträder på kallelse av sektionsstyrelsen.
\para Rätt att hos sektionsstyrelsen begära utlysande av sektionsmöte tillkommer ledamot i sektionsstyrelsen, inspektor, kårens inspektor, kårstyrelsen, sektionens revisorer eller minst 25 datateknologer. Sådant möte ska hållas inom tio läsdagar.
\para Sektionsmöte ska utlysas genom att kallelse anslås minst fem läsdagar i förväg.
\para Kallelse till sektionsmöte ska innehålla samtliga inkomna motioner, propositioner och bemötanden.
\para[Åligganden] Senast dagen före verksamhetsårets början skall följande behandlas på sektionsmöte:
\begin{itemize}
\item Omfördelning av sektionens och dess organs tillgångar.
\item Val av sektionsstyrelse.
\item Val av revisorer.
\item Val av inspektor om så är aktuellt.
\item Sektionsavgift för de två kommande terminerna.
\item Fastställande av preliminär budget för nästkommande verksamhetsår.
\item Fastställande av arvode för nästkommande sektionsordförande och sektionskassör.
\end{itemize}
\para Senast sex månader efter verksamhetsårets början skall följande behandlas på sektionsmöte:
\begin{itemize}
\item Sektionens års- och revisionsberättelse för föregående verksamhetsår.
\item Beslut om ansvarsfrihet.
\item Fastställande av budget för innevarande verksamhetsår.
\end{itemize}
\para[Beslutförhet] Sektionsmötet är beslutsmässigt om mötet är behörigt utlyst enligt detta kapitel.
\stycke
Om färre än 40 datateknologer är närvarande då beslut ska fattas, kan detta endast ske om ingen yrkar på bordläggning.
\stycke Detsamma gäller beslut i frågor som ej har varit anslagna fem läsdagar i förväg.
\para[Motion] Datateknolog som önskar ta upp fråga på föredragningslistan skall anmäla detta skriftligen till sektionsstyrelsen och Talhenspresidiet senast sju läsdagar före sektionsmöte.
\para[Överklagande] Beslut av sektionsmötet som strider mot kårens eller sektionens stadga, reglemente, ekonomiska reglemente eller policy får undanröjas av kårfullmäktige.
\stycke Sådant beslut ska tas upp till prövning om det begärs av en kårmedlem då det rör kårens stadga, eller datateknolog då det rör sektionens stadga, reglemente, ekonomiska reglemente eller policy.
\para[Omröstning] Röstning med fullmakt får ej ske.
\para Omröstning skall ske öppet, om ej sluten votering begärs.
\para Vid lika röstutfall äger mötesordförande utslagsröst, utom vid personval då lotten avgör.
\para Då flera förslag ställs mot varandra skall röstningsförfarandet fastslås innan omröstning påbörjas.
\para Alla frågor som behandlas på sektionsmötet avgörs med enkel röstövervikt om inget annat anges i stadgan. Nedlagda röster räknas ej.
\para[Rättigheter] Närvarorätt tillkommer alla medlemmar, kårledningsledamöter, inspektor, kårens inspektor, revisorer samt av mötet adjungerade icke-medlemmar.
\para Yttranderätt tillkommer datateknolog, hedersmedlem, kårledningsledamöter, inspektor, kårens inspektor, revisorer samt av mötet adjungerade icke-medlemmar.
\para Förslagsrätt tillkommer datateknolog, inspektor samt av mötet adjungerade icke-medlemmar.
\para Rösträtt tillkommer datateknolog.
\para[Persondiskussion] Undantag för samtliga rättigheter är då individ är ämnet av en utlyst persondiskussion. 
\para[Protokoll] Sektionsmötesprotokoll skall justeras av två av mötet valda justeringsmän. Justerat protokoll ska anslås senast tio läsdagar efter mötet.

%Valberedning%%%%%%%%%%%%%%%%%%%%%%%%%%%%%%%%%%%%%%%%%%%%%%%%%%%%%%%%%
\section{Valberedningen och personval}
\para[Sammansättning] sektionens valberedning skall väljas av sektionsmötet på förslag av sektionsstyrelsen.
\para Valberedningen skall bestå av minst fyra ledamöter varav en sammankallande.
\para Valberedningen har till sitt förfogande respektive sektionsorgan för rådgivning.
\para Ledamot i valberedningen behöver ej vara medlem.
\para[Beslutsförhet] Valberedningen är beslutsför då en enkel (1⁄2) majoritet av dess ledamöter är närvarande.
\para Den sammankallande ledamoten har utslagsröst vid lika röstetal.
\para[Åligganden] Valberedningen ansvarar för samtliga nomineringar till förtroendeposter på sektionen.
\stycke Undantag för detta är nomineringar till valberedningen.
\para Valberedningens nomineringar skall anslås i kallelsen sju dagar före valförättandet.
\para Ledamot i valberedningen har ansvar för en nominering tills dess att mandaten de nominerar till blivit ansvarsbefriad. 

%Sektionsstyrelsen%%%%%%%%%%%%%%%%%%%%%%%%%%%%%%%%%%%%%%%%%%%%%%%%%%%%%
\section{Sektionsstyrelsen}
\para[Definition och ansvar] Sektionsstyrelsen handhar i överensstämmelse med denna stadga, befintligt reglemente, befintligt ekonomiskt reglemente samt av sektionsmötet fattade beslut den verkställande ledningen av sektionens verksamhet.
\para Sektionsstyrelsen består av:
\begin{itemize}\setlength\itemsep{0em}
\item Sektionsordförande,
\item Vice sektionsordförande,
\item Sektionskassör,
\item Sektionssekreterare,
\item Studerandearbetsmiljöombud (SAMO),
\item Ordföranden för sektionens kommittéer och studienämnd.
\end{itemize}
\para[Åligganden] Det åligger sektionsstyrelsen att tillsammans:
\begin{itemize}\setlength\itemsep{0em}
\item leda och samordna sektionens verksamhet i enlighet med sektionens beslut och mål,
\item verkställa beslut fattade av sektionsmöten och säkerställa att dessa efterlevs,
\item förbereda ärenden och handlingar inför sektionsmöten och kalla till sektionsmöten enligt stadgan,
\item förvalta sektionens ekonomi och medel ansvarsfullt samt utarbeta förslag till budget och verksamhetsplan,
\item representera sektionen gentemot Chalmers, Studentkåren och externa parter,
\item övervaka och stödja arbetet i sektionens kommittéer och övriga organ, 
\item förvalta sektionens interna och externa kommunikation samt korrespondens, 
\item säkerställa att sektionens styrdokument hålls uppdaterade och att de efterlevs,
\item planera för långsiktig utveckling av sektionens verksamhet och underlätta överlämning till nästkommande förtroendevalda.
\item utse representant från styrelsen till DatE-ITs styrelse
\end{itemize}
\para[Firmateckning] Sektionens firma tecknas, förutom av sektionsstyrelsen, av sektionsordförande och sektionskassör i förening.
\stycke Sektionsstyrelsen må för särskilt fall utse speciell firmatecknare.
\para[Styrelsemöte] Sektionsstyrelsen sammanträder på kallelse av ordförande eller vice ordförande i sektionsstyrelsen.
\para Medlem av sektionsstyrelsen äger rätt att hos vice ordförande i sektionsstyrelsen begära utlysande av styrelsemöte.
\stycke Sådant möte skall hållas inom 5 läsdagar.
\para Sektionsstyrelsen är beslutsmässigt när minst 50\% av medlemmarna är närvarande. Ordförande eller vice ordförande skall närvara.
\para Protokoll ska föras vid styrelsemöte, justeras av två medlemmar av sektionsstyrelsen och anslås senast fem läsdagar efter mötet.
\para[Överklagande] Beslut av sektionsstyrelsen som strider mot kårens eller sektionens stadga, reglemente, ekonomiska reglemente samt policy får undanröjas av kårens fullmäktige.
\stycke Sådant beslut skall tas upp till prövning om det begärs av en kårmedlem då det rör kårens stadga, eller av sektionsmedlem då det rör sektionens stadga, reglemente, ekonomiska reglemente eller policy.

%Kommittéer%%%%%%%%%%%%%%%%%%%%%%%%%%%%%%%%%%%%%%%%%%%%%%%%%%%%
\section{Kommittéer}
\para[Definition] Kommitté ska ha ordförande, kassör och ett i reglementet fastställt antal förtroendeposter.
\para Kommitté ska verka för sektionens bästa och ha en i reglementet fastställd uppgift.
\para Samtliga poster tillsätts av sektionsmötet, på nominering av Valberedningen.
\para Kommitté äger rätt att i namn och emblem använda sektionens namn och dess symboler i enlighet med Chalmers Studentkårs policyer.
\para Kommitté är skyldig att rätta sig efter sektionens stadga, reglemente, ekonomiska reglemente och övriga fattade beslut.
\para Sektionskommittéernas verksamhet och ekonomi granskas av sektionens revisorer.
\para Sektionens kommittéer förtecknas i reglementet.

\section{Utskott}
\para Utskott ska ha ordförande och ett i reglementet fastställt antal förtroendeposter.
\para Utskott ska verka för sektionens bästa och ha en i reglementet fastställd uppgift.
\para Samtliga poster tillsätts av sektionsmötet om inte annat bestäms i reglementet. 
\para Utskott äger rätt att i namn och emblem använda sektionens namn och dess symboler i enlighet med Chalmers Studentkårs policyer.
\para Utskott är skyldig att rätta sig efter sektionens stadga, reglemente, ekonomiska reglemente och övriga fattade beslut.
\para Mandatperioden för sektionens utskott är 1:a november - 31:a oktober
\para Sektionens permanenta utskott förtecknas i reglementet.
\para[Temporära utskott] Utöver detta äger sektionsstyrelsen rätt att starta upp och avveckla \textit{Temporära utskott}, i folkmun \textit{Arbetsgrupper}.
\para Temporära utskott skall avvecklas eller regleras i reglementet senast två kalenderår efter uppstart.


%Intresseföreningar%%%%%%%%%%%%%%%%%%%%%%%%%%%%%%%%%%%%%%%%%%%%%%%%%%%%
\section{Intresseföreningar}
\para[Definition]
Intresseförening är en sammanslutning av sektionens medlemmar och stödmedlemmar med ett gemensamt intresse. Intresseföreningen ska ha en styrelse bestående av föreningens medlemmar. Ordförande i föreningsstyrelsen väljs av sektionsmötet.

\para Varje sektionsmedlem skall ha rätt till medlemskap. Föreningsmedlem som motverkar föreningens syften kan dock uteslutas av föreningsstyrelsen.
\para[Syfte] Intresseförening skall ha en av sektionsstyrelsen godkänd stadga. Dessutom är föreningen skyldig att rapportera till sektionsstyelsen då deras stadga förändrats.
\para Intresseförening skall verka för sektionens bästa och ha ett syfte i sin egen stadga.
\para sektionens intresseföreningar är de i reglemente förtecknade.
\para[Rättigheter]Intresseförening äger rätt att i namn och emblem använda sektionens namn och symboler.
\para[Skyldigheter] Intresseförening är skyldig att känna till och rätta sig efter sektionens stadgar, reglemente, ekonomiska reglemente, policies, övriga handlingar och beslut.
\para[Ekonomi] Intresseföreningens ekonomi skall ligga under sektionen.
\para sektionens revisorer har rätt att granska föreningens verksamhet och ekonomi. 

%sektionens identitet %%%%%%%%%%%%%%%%%%%%%%%%%%%%%%%%%%%%%%%
\section{Sektionens identitet}
\para[Skyddshelgon] sektionens skyddshelgon är Hacke Hackspett.
\para[Sektionsfärg] sektionens färg är orange.
\para[Sektionslokal] sektionens lokal benäms Basen.

%Protokoll och anslagning%%%%%%%%%%%%%%%%%%%%%%%%%%%%%%%%%%%%%%%%%%%%%%
\section{Protokoll och anslagning}
\para[Allmänt] Protokoll som föres i sektionens olika organ skall innehålla anteckningar om ärendenas art, samtliga ställda och ej återtagna yrkanden, beslut samt särskilda yttranden och reservationer.
\para[Anslagning] Meddelanden och beslut är behörigt anslagna då de anslås på sektionens officiella anslagstavla. sektionens officiella anslagstavla definieras i reglementet.

%Revision och ansvarsfrihet%%%%%%%%%%%%%%%%%%%%%%%%%%%%%%%%%%%%%%%%%%%%
\section{Revision och ansvarsfrihet}
\para[Revisorer] Sektionsmötet utser 2–4 lekmannarevisorer med uppgift att granska sektionens verksamhet och ekonomi under verksamhetsåret.
\stycke sektionens revisorer kan ej inneha annan förtroendepost på sektionen under sitt verksamhetsår.
\stycke Räkenskaper och övriga handlingar skall tillställas revisorerna senast 15 läsdagar före sektionsmöte.
\para[Åligganden] Det åligger revisorerna att skicka in revisionsberättelser till talhenspresidiet senast 5 läsdagar före ordinarie sektionsmöte.
\stycke Revisionsberättelsen skall innehålla yttrande ifråga om ansvarsfrihet för berörda personer.
\para[Ansvarsfrihet] Ansvarsfrihet är beviljad berörda personer då sektionsmötet fattat beslut om detta.
\stycke Skulle förtroendevald på sektionen med ekonomiskt ansvar avgå före mandatperiodens slut, skall revision företagas.

%Sektionens upplösning%%%%%%%%%%%%%%%%%%%%%%%%%%%%%%%%%%%%%%%%%
\section{Sektionens upplösning}
\para[Beslut om upplösning] sektionen upplöses genom beslut på två på varandra följande sektionsmöten, med minst femton läsdagars mellanrum, med minst 60 eller samtliga medlemmar närvarande. För att beslutet skall vara giltigt krävs att det antas med tre fjärdedelar av antalet röster. Begäran om sektionens upplösning måste lämnas in till styret och talhenspresidiet senast 7 läsdagar innan den första läsningen sker.
\para[Tillgångar och nystart] Om sektionsmötet beslutar att upplösa sektionen skall samtliga dess tillgångar och skulder, som framgår av upprättad balansräkning, i och med upplösningen tillfalla Chalmers studentkår att förvalta tills dess att en ny förening eller sektion bildas som representerar studerande på utbildningsprogrammet för Datateknik, Chalmers.
%Ändrings- och tolkningsfrågor%%%%%%%%%%%%%%%%%%%%%%%%%%%%%%%%%%%%%%%%%
\section{Ändrings- och tolkningsfrågor}
\para[Stadgeändringar] Ändring av denna stadga kan endast göras av sektionsmötet. För att vara giltig måste ändringen antas med två tredjedelar av antalet röster vid två på varandra följande sektionsmöten, varav minst ett ordinarie, med minst tio läsdagars mellanrum.
\stycke Ändring av eller tillägg till denna stadga skall godkännas av kårstyrelsen.
\para[Reglementesändring] Ändring av eller tillägg till reglementet eller det ekonomiska reglementet kan endast göras av sektionsmötet. För att vara giltig måste ändringen antas med två tredjedelar av antalet röster.
\para[Ändring eller tolkning av DatE-ITs styrdokument] Ändringar i, och tolkning av DatE-ITs styrdokument görs enligt den process som definieras i DatE-ITs styrdokument.
\para[Tolkningstvist] Uppstår tolkningstvist om dessa stadgars tolkning, tolkas stadgan av inspektor för avgörande. Om sådan ej finns avgörs frågan av Chalmers studentkårs inspektor.
\stycke Vid tolkning av reglemente eller ekonomiskt reglemente gäller, tills frågan avgjorts av sektionsmötet, sektionsstyrelsens tolkning.

%Inspektor%%%%%%%%%%%%%%%%%%%%%%%%%%%%%%%%%%%%%%%%%%%%%%%%%%%%%%%%%%%%%
\section{Inspektor}
\para[Allmänt] Inspektor skall ägna uppmärksamhet åt och stödja sektionens verksamhet. Inspektor skall därvid hållas underrättad om sektionens verksamhet. Inspektor har rätt att ta del av sektionens protokoll och övriga handlingar.
\para[Val] Inspektor väljs av sektionsmötet för en tid av två kalenderår.

%Hedersbetygelser%%%%%%%%%%%%%%%%%%%%%%%%%%%%%%%%%%%%%%%%%%%%%%%%%%%%%%
\section{Hedersbetygelser}
\para[Allmänt] sektionen kan som tack eller hedersbetygelse utdela barspeglar.
\para[Kriterier] För att mottagare skall anses värdig att mottaga en barspegel bör något av nedanstående kriterium vara uppfyllda:
\begin{itemize}
\item ha gjort sektionen en betydande tjänst
\item ha gjort sektionen en betydande björntjänst
\item vara monark och fylla jämt
\end{itemize}
\para[Införskaffande] Det åligger sektionsstyrelsen att tillse att erforderlig mängd barspeglar finnes.

\section{TO-DO och to-consider:}
\begin{itemize}
  \item Definiera vad en persondiskussion är och vem som har rätt att kalla till en sådan. (SAMO i alla fall och mötesordförande då det finns kandidater som ej har valberetts?)
  \item Personinval ska alltid vara slutna då det finns fler än 1 kandidater till en post.
  \item Är Ranked-choice voting (RCV) något att kolla på? Borde testas först, annars är detta väldigt mycket bättre än att skapa konstellationer och rösta.
  \item Instruktioner till organ istället för vårt gigantiska reglemente?
  \item Standardisera överlämningar?
  \item Skydda mer av sektionsmötets process enligt stadgan. För mycket lämnas upp till THPs vilja och tolkning atm.
  \item Definiera hur voteringar, slutna voteringar, digitala voteringar ska gå till.
  \item Lekmannarevisorer istället för revisorer. Eller ska vi bara kalla dem för revisorer istället?
  \item Ekonomigrupp istället för flera sektionskassörer? Vi måste få revision och ansvarsfriheter att hanteras regelbundet.
  \item 3/4 majoritet av närvarande och röstberättigade för att ändra reglemente eller 1/2 av närvarande och röstberättigade över två möten för att ändra reglemente?
  \item Ändring av styrdokument bör dokumenteras i respektive appendix.
  \item Förtydliga vem som kan riva upp beslut av vad. Styrelsen kan riva upp styrelsebeslut, sektionsmötet kan riva upp sektionsmötesbeslut, ej ändringar av stadga eller reglemente.
  \item Förtydliga hur styrdokumenten rangordnas och vilka som gäller vid motsägelser mellan dem.
  \item Tillfälliga kommittéer. Styrelsen kan starta och upplösa sådana.
  \item Definiera DNS i stadgan.
  \item Definiera Talhenspresidiet i stadgan.
  \item Representation? Komittéer som är repsbannade bör kunna fråntas sin rätt att bära märket/ovve utan att inskränka på stadgan.
  \item Vill vi skilja på kommittéer med/utan egen kassör? Med = kommitté, utan = utskott?
  \item Vi bör bygga in något form av skydd för sektionskassören. Det är inte rimligt att vi kan starta kommittéer och föreningar utan kassörer och att sektkass får inf med jobb slängt på sig. Att tillsätta fler kassörer är inte heller en hållbar lösning. Det blir otroligt mycket revision med sedan. En gång i tiden talade man om att kassören skulle vara arvoderad.
  \item Kolla över intresseföreningar.
  \item Hantera vakantsättningar och fyllnadsval.
  \item Definiera ordföranderådet.
\end{itemize}
\section{Changelog:}
Ännu inte helt fullständig, men försöker att få med det större.
\begin{itemize}
  \item Delat upp många stora paragrafer i dess olika stycken. Vissa har blivit egna paragrafer.
  \item Mindre kapitel har flyttats, t.ex. stödmedlem, hedersmedlem och avgifter har blivit del av medlemmar.
  \item Definierat ansvarsposter under Organisation.
  \item Definierat företroendeposter under Organisation.
  \item Strykt kapitlet om avsättning i sin helhet.
  \item Lagt till nya paragrafer om Misstroendeförklaring under Organisation.
  \item Lagt till en paragraf om avsägelse under Organisation. Sektmötet måste hantera det, man kan inte bara släppa sina ansvar och säga hejdå.
  \item Tagit bort vissa rättigheter under Medlemmar då de definieras under Sektionsmötet.
  \item Yote den gamla Valberedningen. Den tillför inget nyttigt och fungerar otroligt konstigt om man kollar på hur valberedningar ska fungera. Extremt skumt att man utser sina efterträdare.
  \item Personinval ska ALLTID vara slutna då det finns fler kandidater till en post.
  \item Vi fungerar tydligen inte utan DNS, så vid avsättande av DNS ska ny (interim) väljas omedelbart.
  \item Samma sak med revisorerna.
  \item Kastat om lite i ordningen som vi utövar vår verksamhet. Revisorerna är ju sist, de gör sitt efter att största delen av verksamheten är utförd.
  \item Styrelsen har inte revisorerna eller valberedningen till sitt förfogande. Som sagt, väldigt skumt om styrelsen kan styra över deras revision eller valberedningen till deras efterträdare.
  \item La till kapitlet om intresseföreningar enligt styrets prop. Bör ses över.
  \item La till säte för sektionen under Allmänt.
  \item Döpte om sektionshelgon och sektionsfärg till sektionens identitet.
  \item La till paragraf om sektionslokal under Sektionens identitet.
  \item Ändrat paragrafen om sektionsavgift. Den faställs av kåren.
\end{itemize}
\end{document}
