\section{Aspningspolicy}
Alla kommittéer på sektionen har rätt till att hålla i en aspning. Tanken med aspningen är att visa asparna vad kommitténs arbete går ut på och ge dem en chans att testa på arbetet. Fokus ska ligga på verksamheten. För att se till att alla kommittéer har samma förutsättningar och att alla aspar behandlas på ett värdigt sätt har denna aspningspolicy tagits fram. 

\subsection{Innan Aspningen}
\begin{itemize}
    \item Eftersom det är väldigt många kommittéer som arrangerar aspning samtidigt är det viktigt att sätta sitt aspschema så tidigt som möjligt. Det kommer vara omöjligt att helt undvika dubbelbokningar men se till att undvika det i största mån. 
    \item Innan aspningen börjar är det kul med aspaffischer, dessa kan sättas upp i Basen men styrelsen har rätt att plocka ner aspaffischer som anses vara opassande. Riktlinjer för aspaffischer är
    \begin{itemize}
        \item Alkohol får inte vara fokus av affischen.
        \item Affischer får inte innehålla något form av kränkande material. 
    \end{itemize}
\end{itemize}
\subsubsection{Aspningsplan}
Innan aspningsperioden böjar ska kommittéer skriva ner en aspningsplan som innehåller arr som ska hållas, en beskrivning av dem och en ungefärlig kostnadsberäkning och sedan skicka detta till styrelsen. 


\subsection{Under Aspningen}
\subsubsection{Förhållningsregler}
\begin{itemize}
    \item Det är inte tillåtet att lyfta sin egen kommitté på bekostnad av en annan kommitté. 
    \item Det är inte tillåtet att avråda aspar från att aspa specifika kommittéer. 
    \item Aspningstillfällen skall marknadsföras i enlighet med sektionens kommunikationspolicy, och alla skall uppmuntras lika mycket till att söka eller aspa.
    \item Ingen alkoholhets får förekomma. 
    \item Om alkohol erbjuds vid ett aspningsarrangemang så skall även motsvarande alkoholfria alternativ erbjudas.
    \item Arrangörer har rätt att avvisa aspar som missköter sig från pågående asparrangemang, men endast styrelsen har rätt att stänga av aspar från aspningen i helhet.
    \item Person som arrangerar aspning eller medverkar på aspning av annan anledning än att aspa får ej missbruka sin maktposition på något sätt.
    \item Person som arrangerar aspning ska avstå från att inleda intim relation med asp.
    \item Aspar får inte favoriseras eller särbehandlas.
    \item Aspplagg ska vara roliga att ha på sig och inte vara förnedrande för den som bär det. 
\end{itemize}
\subsubsection{Arrangemang}
\begin{itemize}
    \item Vid varje arrangemang ska det finnas minst två stycken nyktra arrangerande som har ansvar. 
    \item Platsbegränsade arrangemang bör undvikas då alla ska ha möjlighet att delta under aspningen.
    \item Asptillfällen ska vara så billiga som möjligt att delta på, men det är okej för aspar att betala en mindre summa för mat om det skulle behövas. 
    \item Det ska meddelas i förväg ifall mat serveras eller ej under ett asptillfälle. 
    \item Asptillfällen ska ej ha fokus på alkohol, utan ska fokusera på kommitténs verksamhet.
\end{itemize}


\subsection{Efter Aspningen}
Efter aspningen är slut ska alla aspar fortsatt behandlas lika och ingen favorisering får förekomma även efter nominering från valberedningen. 
\subsubsection{Valberedning}
\begin{itemize}
    \item Varje kommitté har rätt att be om valberedning, valberedningens sammansättning beskrivs i reglementet. 
    \item Firande av/festande med nominerad grupp i egenskap av nominering får ej förekomma. 
    \item Det ska vara tydligt för de icke-nominerade att de fortfarande har möjlighet att söka kommittén på sektionsmötet. 
\end{itemize}

