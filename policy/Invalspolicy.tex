

\section{Valberednings- och Invalspolicy}

Denna policy beskriver hur valberedningens process ska genomföras samt hur inval till sektionens kommittéer ska ske. Alla medlemmar har rätt att söka alla förtroendeposter inom sektionen.

\subsection{Valberedning} Alla kommittéer har rätt att begära valberedning från styrelsen inför inval av nya kommittemedlemmar. Valberedningens giltighet definieras i sektionens reglemente. Valberedningen har tystnadsplikt gällande diskussioner och detaljer om de sökande. Endast en sammanfattande bedömning av de nominerade ska presenteras vid invalsmötet.

\subsection{Personinval} Inför inval till en kommitté ska intresserade medlemmar anmäla sitt intresse till sektionsstyrelsen senast 10 dagar före det sektionsmöte där invalet ska ske. Detta gäller även fyllnadsval. Anmälan är dock inte bindande.

En lista med alla de sökande som är intresserade av en förtroendepost ska publiceras i sektionens lokaler senast 9 dagar innan det aktuella mötet.

En sektionsmedlem har fortfarande rätt att söka en post även om denne inte anmält sig i förväg. Vid invalsmötet ska sektionsstyrelsen dock tydligt klargöra vilka av de sökande som har anmält sig inom den angivna tidsfristen.

