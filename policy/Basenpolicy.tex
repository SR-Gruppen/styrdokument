\section{Basenpolicy}
Alla medlemmar i Datateknologsektionen har fri tillgång till sektionslokalen Basen. Därför har en policy för användande av Basen tagits fram.

\subsection{Nyttjande av lokalen}
Vid nyttjande av lokalen råder skötsel-, ordning- och städbestämmelser.

\subsubsection{Skyldigheter}
Medlem skall
\begin{itemize}
    \item följa sektionens och kårens styrdokument, samt svensk lag.
    \item Städa efter sig själv.
    \item ej störa andra.
    \item ej avlägsna objekt från Basen utan styrelsens godkännande.
    \item ej förvara stora och klumpiga saker i Basen eller dess närhet.
    \item medverka i Nollstäd.
    \item ansvara för sina gäster.
\end{itemize}

\subsubsection{Undantag}
\begin{itemize}
    \item Vid arrangemang begränsas ytan för gemene teknolog till köket. Vid vissa arrangemang är även vistelse i köket förbjudet, exempelvis under serveringstillstånd.
    \item Under längre skoluppehåll är Basen avstängd, då resurser inte finns för att städa den.
    \item Delar av, eller hela Basen kan efter styrelsebeslut stängas ner av andra anledningar.
\end{itemize}

\subsection{Fest i lokalen}
Alla fester i Basen skall följa nedanstående regler. Ansvariga har rätt till att avvisa folk från lokalen som inte följer dessa regler. Datateknologer har dock alltid rätt att få tillgång till Basens kök.
\subsubsection{Regler för fest i Basen}
\begin{itemize}
    \item En ansvarig ska finnas på plats, som alltid hålla sig vid sina sinnens fulla bruk och ha uppsikt över hela lokalen och dess omgivning.
    \item Man får ej bruka alkohol i Basen mellan 03:00 och 17:00 på helgfria vardagar.
    \item Det är bara sektionsmedlemmar som kan vara ansvariga.
    \item Köket skall vara fritt från fest.
    \item Brandgränsen skall alltid följas(70 personer i \textbf{hela} lokalen).
    \item Efter 22:00 ska fönsterna vara stängda om nästkommande dag är en vardag.
\end{itemize}

\subsubsection{Paxa basen för arr}
För att utföra ett planerat arr i Basen så ska man skicka in en in paxning via Basenpax formuläret. Beslut om paxningar tas upp nästkommande styrelsemöte.
Efter godkännade från styrelsen så ska arrangemanget aktivitetanmälas på studentportalen. Där ska även den ansvariga skrivas upp.

\subsubsection{Spontan fest}
Om du öppnar upp för fest i basen ska detta anmälas direkt vilket man gör via fest@dtek.se.
I mailet anger man vem som står som ansvarig. Om ni känner er osäkra på om en anmälan
krävs, gör det hellre en gång för mycket än för lite då Styrelsen och DRust vill vilka som
vistats i Basen i fall något händer.

\subsubsection{Överlåta festen}
Om du öppnar basen är du första ansvarig att se till att det blir städat. Om du överlåter
festen till någon annan måste personen som tar över ansvaret skicka in ett nytt mail till
fest@dtek.se där den anger att den tar över.

\subsubsection{Städ}
Om man har arrangerat i Basen så ska det städas enligt de instruktioner som finns på
arrstäd-lapparna hängandes i Basens städskrubb. Ifall att någon annan varit i köket och
har lagat mat eller arrangerat skall detta anges, annars är det ert ansvar att det är städat.
Ett mail ska skickas till fest@dtek.se där man säger till att Basen stängs ner, samt bifogar
bilder från Puben, Köket, Hallen och Flipperrummet. Dessa är till för att skydda dig som
ansvarig från skuld. Städningen ska vara avklarade innan 08:00 följande morgon. Om det
inte är städat innan dess eller om DRust alternativt Styret anser att du inte har städat
eller fyllt i hela listan kommer städningen inte att godkännas och kan leda till påföljder.

\subsection{Åtgärder}
Om några regler bryts kommer åtgärder att vidtas. Du som ansvarig kommer att bli med-
delad att ärendet kommer lyftas på ett styrelsemöte. Styrelsen har rätt att bestämma vilka
åtgärder som kommer tas.
