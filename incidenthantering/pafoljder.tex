\section{Påföljder}
Nedan listas vilken typ av åtgärder styrelsen kan använda efter det att en incident har skett, utan inbördes ordning. Dessa måste tidsbestämmas, undantag gäller ``Rött kort'' där en städuppgift som tilldelas av DRust ska utföras för att bli av med straffet. Ett straff får inte tidsbestämmas till över ett år.

\subsection{Definitioner av följder}
\begin{itemize}
    \item \textbf{Varning}\\
    En varning där personen tydligt blir tillsagd vad personen gjort fel. Varningen är tidsbegränsad så länge det anses vara passande då varningen utfärdas och sparas internt under tiden. Detta för att använda som underlag i de fall personen är involverad i ytterligare incidenter. Personen ska informeras om hur länge all information kopplad till personen sparas. 

    \item \textbf{Representationsförbud}\\
    Innebär att den rapporterade inte får lov att representera sektionen i något sammanhang. Personen får inte lov att t.ex. bära kläder kopplade till sektionen eller vara funktionär i sådana miljöer där personen syns. Att vara funktionär i bakgrunden, t.ex. laga mat för ett arrangemang där person i fråga inte syns utåt är fortfarande accepterat.
    \item \textbf{Förlorad åtkomst till teknologsektionens förråd}\\
    Att förlora åtkomsten till förråd kan endast ges till de i kommittéer med tillgång till förråd och innebär att personen inte längre får vistas där. Personen i fråga kan även bli av med sin nyckel eller access om styrelsen anser det lämpligt.
    \item \textbf{Förlorad åtkomst till teknologsektionens lokaler}\\
    Personen får inte längre vistas i någon av Datatekonologsektionens lokaler. Detta medför indragen access till Basen. Dock får personen i fråga fortfarande delta i teknologsektionens arrangemang även om de sker i teknologsektionens lokaler. 
    \item \textbf{Avstängning från sektionens arrangemang}\\
    Innebär att vederbörande inte får delta under något av teknologsektionens arrangemang.
    \item \textbf{``Rött kort''}\\
    En mildare version av "förlorad åtkomst till teknologsektionens lokaler" som börjar verka först två läsveckor efter styrelsens beslut. Detta straffet avslutas efter att den berörda utfört en städuppgift tilldelad av DRust.
%    \item \textbf{Avsättning}\\
%    Att bli avsatt innebär att personen i fråga förlorar sin post i relevanta kommitteer och denna post vakantsätts. Personen ska lämna tillbaka ev. nycklar, bli fråntagen åtkomst till ev. förråd och får inte längre representera teknologsektionen.  
\end{itemize}

\subsection{Riktlinjer vid policyöverträdelser}
\begin{itemize}
    \item\textbf{Varning}\\
    Kan användas vid incidenten är av mildare grad eller i övrigt anses vara nog. Varning är endast lämpligt då den rapporterade inte blivit varnad tidigare. 
    \item \textbf{Representationsförbud}\\
    Kan användas i situationer där den rapporterade bär representationsplagg vid överträdelsen. Representationsförbud kan också vara aktuellt då personen anses representera sektionen på ett negativt sätt. 
    \item \textbf{Förlorad åtkomst till teknologsektionens förråd}\\
    Kan användas i situationer där överträdelsen sker i samband med att personen i fråga haft ansvar för lokalen och/eller befunnit sig i lokalen. 
    Kan användas då incidenten på något sätt kan kopplas till sektionens förråd. T.ex. att personen befann sig där vid incidenten eller missbrukat sitt ansvar över förrådet. 
    \item \textbf{Förlorad åtkomst till teknologsektionens lokaler}\\
    Kan tillämpas då personen skadat och/eller smutsat ner någon av sektionens lokaler. Det kan även användas om personen stått som ansvarig under t.ex. ett arrangemang. Kan också användas i kombination med avstängning från arrangemang för förstärkt straff. 
    \item \textbf{Avstängning från teknologektionens arrangemang}\\
    Kan vara lämpligt då incidenten skett under ett av sektionens arrangemang. T.ex. någon form av kränkning eller annat som skapat obehag hos andra deltagare. 
    \item \textbf{``Rött kort''}\\
    Bör tilldelas Nollan vid missat nollanstäd men undantag att Nollan fyllt i en lämplig ursäkt för att missa städet. Är också lämpligt om andra sektionsmedlemmar inte städat tillräckligt efter arrangemang i sektionens lokaler. 
%    \item \textbf{Avsättning}\\
%    Kan inte ges av styrelsen utan ska tas till sektionsmötet. För att ta ärendet till sektionsmötet krävs två tredjedelars majoritet enligt sektionens stadga. I första hand ska styrelsen be personen att avgå och på så sätt behöver ärendet inte tas till sektionsmötet. Praxis bör vara att stänga av från lokaler och arrangemang under tiden till nästa sektionsmöte. Styrelsen bör rådfråga kårledningen i ärenden som gäller avsättning.
\end{itemize}