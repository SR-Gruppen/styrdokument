\section{Definitioner}
En incident kan sträcka sig från en liten händelse till en mycket allvarlig situation. Beroende på incidentens karaktär kommer således olika åtgärder tas. Nedan följer definitioner av några olika kategorier av incidenter som kan inträffa.

\subsection{Övertramp av styrdokument}
Kan vara om en sektionsmedlem eller kommitté bryter mot sektionens och kårens regler, stadgar, policyer eller dispositionsavtal. Exempel på detta är om en medlem inte gjort sitt nollanstäd, städat dåligt efter arrangemang, inte festanmält arrangemang eller liknande.

\subsection{Trakasserier och kränkande handlingar}
Trakasserier är ett agerande som kränker någons värdighet och/eller kan men måste inte kunna kopplas till någon av de sju diskrimineringsgrunderna. Det är alltid upp till personen som blivit utsatt att avgöra om en handling var trakasserande. Det kan vara allt från våldsamma handlingar, oönskade närmanden eller en taskig kommentar. 

\subsection{Gruppdynamikproblem}
Problematik inom eller mellan grupper, som förvärrar gruppens arbete eller medlemmarnas mående. Det kan bero på organisatoriska problem eller beteenden hos individer i gruppen.

\subsection{Skada av material eller person}
Allvarsnivån av dessa incidenter kan variera mycket. Exempel på incidenter som passar in här kan vara inbrott, stölder, vattenläckage eller personskada.

\subsection{Dödsfall}
Detta kan vara att sektionsstyrelsen hör ett rykte om att en kårmedlem har avlidit, att högskolan informerar styrelsen om det eller att ett dödsfall händer i anknytning till sektionens eller kårens verksamhet.

\subsection{Disciplinärende och straff}
Varken Datateknologsektionen eller Chalmers Studentkår hanterar
disciplinärenden. Disciplinärenden, såsom fusk på tentor, kränkningar, trakasserier och dylikt som allvarligt strider mot Chalmers policies eller som
grovt kan skada Chalmers rykte hanteras också av Chalmers tekniska högskola.