\section{Att tänka på vid hantering av incidenter}
Det viktigaste att tänka på vid möten som behandlar incidenter, oavsett om det är med den som blivit rapporterad, den som rapporterat eller den som blivit påverkad av incidenten, är att alla känner sig hörda.

Punkter att ha i åtanke vid kontakt med inblandade angående skedd incident:
\begin{itemize}
    \item Fler än nödvändigt bör ej närvara på möten med inblandade för att skapa en lugn och trygg stämning.
    \item Var tydlig med varför ett möte har tillkallats och huruvida anteckningar förs.
    \item Be personen berätta sin sida av händelsen utan att hen avbryts
    \begin{itemize}
        \item Om det finns frågor till den inblandade bör dom i största möjliga grad vara förberedda innan mötet
    \end{itemize}
    \item Förklara vad som kommer att hända efter mötet och hänvisa till vem den inblandade ska vända sig till om hen har frågor.
    \item Fråga om den inblandade har några frågor
    \item Efter mötet bör de som närvarat (förutom den inblandade) reflektera över vad som sagts och diskutera vad som bör göras framöver
\end{itemize}

\subsection{Att tänka på i möte med någon som är utsatt}
\begin{itemize}
    \item Det är okej att ställa frågor om hur andra kan ha upplevt situationen men glöm då inte att de endast vet säkert hur de själva känner.
    \item Var tydlig med att du bryr dig, kan även vara bra med pauser om det behövs.
    \item Hänvisa till ytterligare resurser att få hjälp ifall behov av stöd finns från andra än SAMO och styrelsen.
    \item Om personen öppnar upp om saker som inte hör till situationen i sig, lyssna men var tydlig med att du inte har professionell utbildning och att det finns andra de kan prata med.
\end{itemize}

\subsection{Att tänka på i möte med någon rapporterad}
\begin{itemize}
    \item Var inte attackerande i bemötandet och kom ihåg att det finns olika sidor av händelser. Det kan kännas väldigt jobbigt som anklagad och inte haft chans att berätta sin sida.
    \item Var tydliga med vilka konsekvenser som kan komma att ges och hur processen som avgör vilka om några som kommer att ges fungerar.
    \item Försök vara snabba med att bestämma vilka om några konsekvenser som kommer att ges så att den anklagade slipper vänta.
    \item Även den anklagade kan må dåligt så erbjud den utsatte samma stöd som för den utsatte eller likvärdig som är mer lämplig för den anklagades situation.
    \item Om den anklagade blir aggressiv eller arg har ni alltid känna atta ni kan avbryta mötet och kalla på hjälp. Försök lugna personen även om det inte leder till ett fulländat möte där allt blir sagt. Det viktigaste är att du känner dig trygg och om du inte gör det kan du ta hjälp av studentkåren som kan närvara vid eller hålla mötet istället.
\end{itemize}